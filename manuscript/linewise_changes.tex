1-278: reduce?
7.  It is certainly true that most networks are snapshots in time but there is a growing realization – including some nice examples (including those by the authors) that address this issue.  Maybe what you want to say is that we know that networks should not be considered static but we lack the tools – or ability to collect data – or whatever you think is the cause – to correct this situation. 
23.  Delete the “because” (you have one in the sentence already).
30.  There are several conceptual and empirical attempts to consider detection probability – why aren’t these cited (work by Bartomeus comes to mind).
36.  I certainly see value in the approach and the general ideas of these authors in particular but they should acknowledge the work that has been done by themselves and others!
36ff The authors should note the value of weighted metrics in taking account of sampling biases.
45 It is a shame not to give some idea of how the method could be extended (or is this the intention of the authors?). For empirical analysis of networks it is rare to use binary networks given the value of weighted networks for taking account of sampling biases. Indeed, I am much less concerned about whether an interaction never occurs, and much more concerned with the frequency of occurrence.
50 This is a nice and clearly explained description of the problem that many network ecologists ignore, or are unaware of.
62: define formally T
73, 87.  Maybe the idea that some uncertainty is “inevitable” should not be repeated in multiple sections.  Either find a different way of saying it – or state it in your introductory paragraph and then don’t keep repeating.
76ff Different interactions occur at different temporal scales, so it is important to consider where the role of this study lies. For instance, a pollinator-flower interaction is quick (and even the evidence of it, e.g. pollen on the insect, is not long-lasting) whereas an active gall-former interaction could be present for a couple of months and could be detected for even longer (on senescing leaves).
76ff ‘Process uncertainty’ is not defined. The simplistic example of interaction uncertainty (fish eating a cactus) does not help the reader understand the detail of what is meant by ‘interaction uncertainty’. The two examples of ‘local constraints’ (weather and habitat) are operating on completely different temporal scales (note – it is defined differently in L109). I would regard the issue of weather as much closer to detection uncertainty (e.g. for a pollinator, it would be likely to be interacting if the weather was better), whereas the issue of habitat is closer to the authors’ ‘interaction uncertainty’ (the species don’t interact because they are not, or never, co-occurring in a habitat). Also if the species occur in different habitats then with an appropriate spatial scale of sampling then they do not co-occur – so the issue seems redundant. I suspect that the authors are thinking of a meta- or master network in much of their study, but this is not clear (or I have misunderstood and it needs to be explained better), i.e. the predicted presence of an interaction is not the presence of an interaction at the local site (taking detection and process uncertainty into account) but the ability to assess ‘interaction uncertainty’.
86.  So, similar factors (i.e., trait matching) can be used to detect different kinds of uncertainty.  Should this be stated?  You might also consider a different word that process as “observation models” and “process models” are commonly used terms in Bayesian stats and in this case “process” often considers what you call “interaction”.  There are some interesting to attempts to model co-evolution that consider similar ideas (i.e., a barrier to interaction and then trait-matching; but maybe outside of the scope of this MS). While I appreciate that your naming has merit, I think if your hope is that more ecologists use these methods then the terminology should be consistent across papers/approaches to the greatest extent possible.
89.  In Weinstein and Graham (Ecology Letters) the processes modeled were trait matching (which you place in interaction uncertainty) and abundance (which is perhaps your process uncertainty??).  I find myself quite confused based on how you describe things in this MS (See text above) and how you cite the literature.
89: Seems to contradict two preceding paragraphs
94.  I find this a very odd argument against sampling multiple times.  So, if you sample more you might see more then understand the system less?  The logic does not make sense to me. 

107.  There is a lot of literature in wildlife ecology emphasizing the importance of repeat sampling for estimating detection probability.  Is this just wrong?  If so this does need more explanation in the context of this literature.
121.  I think one of the reason non-Bayesians’ are not comfortable with the approach is that if you put in prior information on something like trait matching and test for trait matching and then discover trait matching is important – what does it mean?  Maybe this is too obvious (I realize there are tons of papers on how to choose priors) but I wonder if it is worth explaining (would a box or something be worthwhile?  I leave it up to the authors/editor)?  Maybe the issue could be acknowledged and an appropriate paper cited?
130.  The ideas here are largely a repeat from those above.  Further, while it may be true that we can sample too much – I think it is dangerous to suggest we should sample less – do the authors think that sampling across most network studies is sufficient (the opposite is stated in the discussion)?  I wonder if the argument is a bit more statistical than biological.
136 This is a completely different issue, that you do not address in the paper.
142.  But shouldn’t you model the probability of two species co-occurring?  The authors have a paper doing this…. Is that not a good idea?
144.  I am much more comfortable with this statement than how sampling effort should be considered and I would suggest moving this up and then explaining how even if we sample well we need to consider uncertainty for the reasons you mention in the previous section.  Please note that you make the same point in your discussion on line 338.
147 Your Bayesian approach seems important, especially the prior information, and I would have appreciated it being explained more simply.
148.  Please note that a paper with a very similar aim was recently published (Graham and Weinstein).  This does not to diminish the value of this study in any way, as multiple perspectives on quantitative solutions in network ecology are urgently needed….. but the other work should be considered. More generally, the introduction does not really do justice to what has been done already – even the authors own work.  The problem in my view is that the community using better statistics – of which the authors are very important members – has not managed to communicate to most people studying networks and better methods are simply not being adopted.  The current paper is a very nice attempt to show the broader community that the Bayesian approach is not so complex and has many advantages.  I think this is a very important message and is what makes this paper an important contribution.  Further, the approach to parse out different types of uncertainty and suggestions for how to come-up with informative priors – is very useful.  The theory/method is well developed overall and provides an example – which is great.
149: since the authors consider the interactions are independent, there is morally no network in the approach (except the result)
155ff I think it would be helpful for this section to be integrated with the previous section. It relies on the reader putting quite a lot of work in to understand the link between the two.
209ff This seems a really important set of ideas, but they are exemplified very briefly and very simplistically in the paper (L262, L310). I would have appreciated a much clearer explanation and example.
235.  Correct redundancy.
240.  I found the justification for the system a bit odd.  Is the goal to show the gaps in sampling or to apply the model described thus far to consider the 3 different types of uncertainty outlined?  At this point the reader is confused about the goal…
241 I wonder if you could phrase this more positively as ‘solutions’ rather than just ‘difficulties’?
244: this is weird to mention that the analysis has been done on another dataset but that the results will not be discussed… What is the interest for the reader?
249.  The statement about training data is redundant with what is stated above – adjust writing.
252: not clear. Why this choice?
259: not clear
278: Add citations
278: the authors investigate the consequence of uncertainity on network metrics. This question is of peculiar interest, but here it is restricted to the connectance and nestedness. Why these two metrics? Why not others? What is the impact of integrating incertainity into the metrics computation? (we guess the answers but we could expect the authors develop these points).
283 This seems a very important point and should be explained/expanded more clearly.
288.  I am pretty sure this idea is in the literature several times and should be cited.  Isn’t this the same as drawing a probability of interaction from a distribution where the distribution of each interaction is an output of the Bayesian method?
291 I have wondered about this – does assuming a probability of 1 for observed interactions create a bias? The observed interactions are a stochastic set of the possible interactions, so by treating observed interactions as 1 and other (equally likely?) unobserved interactions as <1 will surely bias the network metrics. I’m not sure of the solution and would welcome thoughts on this in the paper.
291: explain better the simulation procedure
295 to 302.  I would think that this second step (i.e., filtered networks) is to explore sampling not to determine how networks will be influenced by uncertainty?  Do you need the filtering step to evaluate uncertainty (as written it seems that this is the case)?
298 These seem arbitrary. It is not clear what is filtered, or its justification – is it the unique interactions, unique interaction per site or samples? How do these compare to typical sampling effort?
306: given the shape of the distribution (exponential), the mean is not an appropriate indicator.
319.  It is clear that these are large samples but if interaction among co-occurring species was greater than you need lower samples – correct?  Should this be made clear?  Otherwise, we maybe can just never get enough data!
321 This is a really important and interesting set of results that are worthy of greater discussion. The issue of metrics from sampled networks is important, but gets lost in the paper. Please expand upon this.
327: what does it mean? How is it possible?
338 to 342.  Throughout I am confused if the authors suggest more sampling is good or bad…  At the end of the paragraph it seems it isn’t so great to sample more because you end up with these annoying “0” values…??????
343-351: difficult to relate results to discussion
352 – 358.  Citations are needed in this section as this has been suggested before.
361.  I find the statement that 30-50 individuals need to be evaluated a bit strong.  This statement is based on this study – are all systems like the gall system?  If species are specialized or bound to interact for some other reason (i.e., co-occurring when there are few other resources) wouldn’t you need to observe fewer individuals?  The point of the method is that you can estimate how many individuals are needed…. This is great!  But given that this can be estimated from any given system why give a value from one system and state that is what is required?  It defeats the point of the very nice method proposed…??
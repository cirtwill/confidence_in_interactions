
\documentclass[12pt]{article} 
\usepackage{amsmath} 
\usepackage[dvips]{graphicx}
\usepackage{multirow} 
\usepackage{geometry} 
\usepackage{pdflscape}
\usepackage[labelfont=bf]{caption} 
\usepackage{setspace}
\usepackage[running]{lineno} 
% \usepackage[numbers,sort]{natbib}
\usepackage[round]{natbib} 
\usepackage{array}

\newcommand{\methods}{\textit{Materials \& Methods}}
\newcommand{\SI}{\textit{Appendix}~}

\topmargin -1.5cm % 0.0cm 
\oddsidemargin 0.0cm % 0.2cm 
\textwidth 6.5in
\textheight 9.0in % 21cm
\footskip 1.0cm % 1.0cm

\usepackage{authblk}

\title{Supplemental information}


\author{Authors}
\date{\small$^1$Department of Physics, Chemistry\\ 
and Biology (IFM)\\ 
Link\"{o}ping University\\
Link\"{o}ping, Sweden\\
% \medskip
% $^\dagger$ Corresponding author:\\
% alyssa.cirtwill@gmail.com\\
% +46 723 158464\\
 }

\renewcommand\Authands{ and }

\begin{document} 
\maketitle 
\raggedright
\setlength{\parindent}{15pt} 
\clearpage

\section*{Appendix S1: Details of \emph{Salix} data collection}

      The \emph{Salix}-galler-parasitoid meta-network dataset consists of a single community type sampled across Europe; willow (\emph{Salix}) species, willow-galling sawflies (Hymenoptera, Tenthredinidae, Nematinae, Euurina), and their natural enemies (hymenopteran parasitoids and coleopteran, lepidopteran, dipteran, and hymenopteran inquilines). The data were collected over 29 years (1982-2010) at 374 unique locations across Europe ranging from Sicily to the Arctic. Including repeated visits to a number of sites, the data set contains 641 site-visits, each of which can be considered as a network in its own right and an independent sample from which to build the meta-network. The meta-network consists of 1,173 different interactions between 52 \emph{Salix} nodes, 92 herbivore nodes, and 126 parasitoid nodes. Interactions were determined by dissecting and rearing gall inhabitants from 165,424 galls. The high spatiotemporal resolution of this network and the unusually high sampling effor implemented at the site level (641 site-visits, which can be considered as repeated samples of the network) makes this dataset particularly well suited for testing this method.




\section*{Appendix S1: Obtaining prior interaction probabilities}

    \subsection*{Creating a binary interaction network}

      Barbour \emph{et al.} sampled 145 branches from 
      \emph{Salix hookeriana} of 26 genotypes. They recorded galls 
      made by four species of Cecidomyid midges on each branch. We 
      transformed these data to a binary genotype-galler network 
      matrix, where an entry $ij$ was 1 if galler $i$ was observed 
      on any branch of genotype $j$ and 0 otherwise. In total, this 
      network contained 75 realised galler-genotype interactions 
      out of a potential 104. One \emph{S. hookeriana} genotype did
      not interact with any gallers and was removed from the network.
      The R code used to extract this 
      network follows:


      \vspace{6pt}
      \emph{
      \noindent \hspace{-4pt} prior\_web\_data=read.csv(``tree\_level\_interaxn\_all\_plants\_traits\_size.csv")\\
      \# Remove extra data on parasitoids, etc.\\
      prior\_web\_data[,31:57]\textless-NULL\\
      prior\_web\_data[,4:26]\textless-NULL\\
      prior\_web\_data[,2]\textless-NULL\\
      \# Build a web with interactions=1 for any galler observed on any genotype\\
      prior\_web=matrix(nrow=26,ncol=4)\\
      for(r in 1:length(levels(prior\_web\_data\$Genotype)))\{\\
      \hspace{4pt}  gen=levels(prior\_web\_data\$Genotype)[r]\\
      \hspace{4pt}  subset=prior\_web\_data[which(prior\_web\_data\$Genotype==gen),]\\
      \hspace{4pt}  for(c in 1:4)\{\\
      \hspace{8pt}    if(sum(subset[,2+c])\textgreater0)\{\\
      \hspace{12pt}      prior\_web[r,c]=1\\
      \hspace{8pt}    \} else \{\\
      \hspace{12pt}      prior\_web[r,c]=0\\
      \hspace{8pt}    \}\}\}
      \# Remove non-interactive genotype
      prior\_web\textless-prior\_web[which(rowSums(prior\_web)\textgreater0),]
      }

      % prior_web_data=read.csv('../../data/Salix_example/tree_level_interaxn_all_plants_traits_size.csv')
      % # Remove extra data on parasitoids, etc.
      % prior_web_data[,31:57]<-NULL
      % prior_web_data[,4:26]<-NULL
      % prior_web_data[,2]<-NULL
      % # Build a web with interactions=1 for any galler observed on any genotype
      % prior_web=matrix(nrow=26,ncol=4)
      % for(r in 1:length(levels(prior_web_data$Genotype))){
      %   gen=levels(prior_web_data$Genotype)[r]
      %   subset=prior_web_data[which(prior_web_data$Genotype==gen),]
      %   for(c in 1:4){
      %     if(sum(subset[,2+c])>0){
      %       prior_web[r,c]=1
      %     } else {
      %       prior_web[r,c]=0
      %     }
      %   }        
      % }
      % prior_web<-prior_web[which(rowSums(prior_web)>0),]
      % write.csv(prior_web,file='binary_prior_web_SG.csv',quote=F)

      \vspace{6pt}


      To extract the galler-parsitoid network from the same study above:


      \vspace{6pt}
      \noindent \emph{
      \hspace{-4pt} prior\_web\_data=read.csv(``tree\_level\_interaxn\_all\_plants\_traits\_size.csv")\\
      prior\_web\_data \textless- prior\_web\_data[,8:26]\\

      \# Build a web with interactions=1 for any parasitoid on any galler\\
      prior\_web=data.frame()\\

      \# each column contains all interactions for a particular parasitoid-galler combination\\
      for(column in 1:ncol(prior\_web\_data))\{\\
      \hspace{-4pt}   \# the column name gives the species which makes the gall, followed by the species emerging from the gall\\
      \hspace{-4pt}   galler\_emerged=strsplit(colnames(prior\_web\_data)[column], "\_")\\
      \hspace{-4pt}   galler = galler\_emerged[[1]][1]\\
      \hspace{-4pt}   emerged = galler\_emerged[[1]][2]\\
        
      \hspace{-4pt}   \# the strength of the interaction is the sum of the column (number of galls across all branches sampled)\\
      \hspace{-4pt}   linkstrength = sum(prior\_web\_data[,column])\\
      \hspace{-4pt}   prior\_web[as.character(galler),as.character(emerged)] \textless- linkstrength\\
      \} 
      \# Make the NAs into 0\\
      prior\_web[is.na(prior\_web)] \textless- 0\\
      \# remove emerging gallers with no parasitoids\\
      prior\_web\_para\_only \textless- prior\_web[,c(2:6,8:9,12)]\\
      \# convert interaction strengths to 1\\
      prior\_web[which(prior\_web\textgreater0)] \textless- 1\\
      write.csv(prior\_web\_para\_only, file=``prior\_web\_para\_only.csv", quote=F)
      }\par

      % % # read in the data
      % prior_web_data=read.csv('../../data/Salix_example/tree_level_interaxn_all_plants_traits_size.csv')
      % # just keep what we need
      % prior_web_data <- prior_web_data[,8:26]

      % # Build a web with interactions=1 for any parasitoid on any galler
      % prior_web=data.frame()

      % # each column contains all interactions for a particular parasitoid-galler combination
      % for(column in 1:ncol(prior_web_data)){
      %   # the column name gives the species which makes the gall, followed by the species emerging from the gall
      %   # they need to be separated 
      %   galler_emerged=strsplit(colnames(prior_web_data)[column], "_")
      %   galler = galler_emerged[[1]][1]
      %   emerged = galler_emerged[[1]][2]
        
      %   # the strength of the interaction is the sum of the column (number of galls across all branches sampled)
      %   linkstrength = sum(prior_web_data[,column])
      %   # put that into a dataframe
      %   prior_web[as.character(galler),as.character(emerged)] <- linkstrength
      % } 

      % # Make the naS into 0
      % prior_web[is.na(prior_web)] <- 0
      % # Make the non-zero elements into 1
      % prior_web[prior_web>0] <- 1
      % # some of the emergers were the gallers, remove those columns if we're only
      % # interested in galler-parasitoid interactions
      % prior_web_para_only <- prior_web[,c(2:6,8:9,12)]
      % write.csv(prior_web_para_only, file="binary_prior_web_para_only.csv", quote=F)


    \subsection*{Degree distributions and interaction probabilities}

      Obtaining degree distributions from the binary interaction 
      network is straightforward. To do this, we simply divide the 
      total number of observed interaction partners for each 
      species (row or column sums) with the number of potential 
      interaction partners (number of column or rows). Degree 
      distributions calculated, we can then calculate the 
      probability of each interaction as the product of the 
      normalised degrees of the two species. For example, the 
      probability of an interaction between two species with 
      normalised degrees of 0.5 (each interacts with half of the 
      available partners) is 0.25. R code used to do this follows:


      \vspace{12pt}
      \emph{
      \noindent \hspace{-4pt} \# Now get the degree distributions from the web\\
      deg\_dist\_Salix=rowSums(prior\_web)/ncol(prior\_web)\\
      \# Remove one genotype that never interacted\\
      deg\_dist\_Salix\textless-deg\_dist\_Salix[which(rowSums(prior\_web)\textgreater0)]\\
      deg\_dist\_galler=colSums(prior\_web)/nrow(prior\_web)\\
      \# Interaction probabilities are the product of plant and galler probabilities\\
      int\_probs=as.numeric(deg\_dist\_galler\%$*$\%t(deg\_dist\_Salix))\\

      \# And for the galler-parasitoid web:\\
      deg\_dist\_parasitoid=rowSums(prior\_web\_para\_only)/ncol(prior\_web\_para\_only)\\
      deg\_dist\_parasitized=colSums(prior\_web\_para\_only)/nrow(prior\_web\_para\_only)\\
      gp\_int\_probs=as.numeric(deg\_dist\_parasitized\%$*$\%t(deg\_dist\_parasitoid))

      }

      SG_web=read.csv('binary_prior_web_SG.csv',row.names=1)
      GP_web=read.csv('binary_prior_web_para_only.csv',row.names=1)

            # Now get the degree distributions from the web
      deg_dist_Salix=rowSums(SG_web)/ncol(SG_web)
      deg_dist_galler=colSums(SG_web)/nrow(SG_web)
      # Interaction probabilities are the product of plant and galler probabilities
      sg_int_probs=as.numeric(deg_dist_galler%*%t(deg_dist_Salix))

      deg_dist_parasitoid=rowSums(GP_web)/ncol(GP_web)
      deg_dist_parasitized=colSums(GP_web)/nrow(GP_web)
      gp_int_probs=as.numeric(deg_dist_parasitized%*%t(deg_dist_parasitoid))


\clearpage


\section*{Appendix S2: Calculating posterior distributions and confidence intervals}

  The following simple functions are implemented in the R language. 
  Priordata is the list of interaction frequencies from the prior 
  data, n is the number of sites with observed co-occurrance of 
  species $i$ and $j$, and k is the number of sites with an 
  observed interaction $ij$. When calculating the prior 
  distribution, both n and k are 0. In the main text we assume 100\%
  confidence in observed interactions and therefore consider only 
  cases where k=0. Only n varies between species pairs.


  To calculate the parameters $\alpha$ and $\beta$ of a prior or 
  posterior distribution:

  \vspace{12pt}
  \emph{
  \noindent \hspace{-4pt}calculate\_parameters\textless-function(priordata,n,k)\{\\
  \hspace{4pt} \# Calculate prior parameters\\
  \hspace{4pt} start=list(shape1=1,shape2=1)
  \hspace{80pt} pars=fitdistr(x=priordata,`beta',start=start,lower=c(0,0))\$estimate\\
  \hspace{4pt} alpha=pars[[1]]\\
  \hspace{4pt} beta=pars[[2]]\\
  \hspace{4pt} \# Update the parameters with data. If n=0 and k=0, no change.\\
  \hspace{4pt} alpha\_prime=alpha+k\\
  \hspace{4pt} beta\_prime=beta+n-k\\
  \hspace{4pt} pars2=c(alpha\_prime,beta\_prime)\\
  \hspace{4pt} return(pars2) \}
  }
  \vspace{12pt}


    To calculate the maximum likelihood estimates of the mean and 
    variance of the probability of interaction $\lambda_{ij}$:

    \vspace{12pt}
    \emph{
    calculate\_distribution\textless-function(pars)\{\\
      \# alpha and beta may be from a prior or posterior distribution\\
      alpha=pars[[1]]\\
      beta=pars[[2]]\\


      \# Calculate the MLE of the mean\\
      mu\_num=alpha\\
      mu\_den=alpha+beta\\
      mu=mu\_num\/mu\_den\\


      \# Calculate the MLE of the variance\\
      sig\_num=alpha\*beta\\
      den1=alpha+beta\\
      den2=den1\*\*2\\
      sig\_den=den1\*den2\\
      sigma2=sig\_num\/sig\_den\\


      return(c(mu,sigma2))\}
    }
  \vspace{12pt}
      

  To calculate a credible interval based on the prior or posterior distribution, for given lower and upper bounds:


  \vspace{12pt}
  \emph{
  credible\_interval\textless-function(pars,p\_lower,p\_upper)\{\\
    alpha=pars[[1]]\\
    beta=pars[[2]]\\
    lowCI=qbeta(p=p\_lower,shape1=alpha,shape2=beta)\\
    highCI=qbeta(p=p\_upper,shape1=alpha,shape2=beta)\\
    return(c(lowCI,highCI))\}
  }


  \vspace{12pt}
  \emph{
  plot\_precision\textless-function(threshold,confidence,pars)\{\\
    alpha=pars[[1]]\\
    beta=pars[[2]]\\
    n=seq(0,100,1)\\
    k=0\\
    cdf=pbeta(threshold,shape1=alpha,shape2=beta+n)\\
    samples=length(which(cdf<confidence))\\
    return(samples) \}
  }


\clearpage

    \bibliographystyle{ecollett} 
    \bibliography{manual_abbrev} % Abbreviate journal titles.


\end{document}



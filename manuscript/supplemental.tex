
\documentclass[12pt]{article} 
\usepackage{amsmath} 
\usepackage[dvips]{graphicx}
\usepackage{multirow} 
\usepackage{geometry} 
\usepackage{pdflscape}
\usepackage[labelfont=bf]{caption} 
\usepackage{setspace}
\usepackage[running]{lineno} 
% \usepackage[numbers,sort]{natbib}
\usepackage[round]{natbib} 
\usepackage{array}

\newcommand{\methods}{\textit{Materials \& Methods}}
\newcommand{\SI}{\textit{Appendix}~}

\topmargin -1.5cm % 0.0cm 
\oddsidemargin 0.0cm % 0.2cm 
\textwidth 6.5in
\textheight 9.0in % 21cm
\footskip 1.0cm % 1.0cm

\usepackage{authblk}

\title{Supplemental information}


\author{Authors}
\date{\small$^1$Department of Physics, Chemistry\\ 
and Biology (IFM)\\ 
Link\"{o}ping University\\
Link\"{o}ping, Sweden\\
% \medskip
% $^\dagger$ Corresponding author:\\
% alyssa.cirtwill@gmail.com\\
% +46 723 158464\\
 }

\renewcommand\Authands{ and }

\begin{document} 
\maketitle 
\raggedright
\setlength{\parindent}{15pt} 
\clearpage

\section*{Appendix S1: Obtaining prior interaction probabilities}

    \subsection*{Creating a binary interaction network}

      Barbour \emph{et al.} sampled 145 branches from 
      \emph{Salix hookeriana} of 26 genotypes. They recorded galls 
      made by four species of Cecidomyid midges on each branch. We 
      transformed these data to a binary genotype-galler network 
      matrix, where an entry $ij$ was 1 if galler $i$ was observed 
      on any branch of genotype $j$ and 0 otherwise. In total, this 
      network contained 75 realized galler-genotype interactions 
      out of a potential 104. The R code used to extract this 
      network follows:


      \vspace{12pt}
      \emph{
      \noindent \hspace{-4pt} prior\_web\_data=read.csv(``tree\_level\_interaxn\_all\_plants\_traits\_size.csv")\\
      \# Remove extra data on parasitoids, etc.\\
      prior\_web\_data[,31:57]\textless-NULL\\
      prior\_web\_data[,4:26]\textless-NULL\\
      prior\_web\_data[,2]\textless-NULL\\
      \# Build a web with interactions=1 for any galler observed on any genotype\\
      prior\_web=matrix(nrow=26,ncol=4)\\
      for(r in 1:length(levels(prior\_web\_data\$Genotype)))\{\\
      \hspace{4pt}  gen=levels(prior\_web\_data\$Genotype)[r]\\
      \hspace{4pt}  subset=prior\_web\_data[which(prior\_web\_data\$Genotype==gen),]\\
      \hspace{4pt}  for(c in 1:4)\{\\
      \hspace{8pt}    if(sum(subset[,2+c])\textgreater0)\{\\
      \hspace{12pt}      prior\_web[r,c]=1\\
      \hspace{8pt}    \} else \{\\
      \hspace{12pt}      prior\_web[r,c]=0\\
      \hspace{8pt}    \}\}\}
      }

      % prior_web_data=read.csv('tree_level_interaxn_all_plants_traits_size.csv')
      % # Remove extra data on parasitoids, etc.
      % prior_web_data[,31:57]<-NULL
      % prior_web_data[,4:26]<-NULL
      % prior_web_data[,2]<-NULL
      % # Build a web with interactions=1 for any galler observed on any genotype
      % prior_web=matrix(nrow=26,ncol=4)
      % for(r in 1:length(levels(prior_web_data$Genotype))){
      %   gen=levels(prior_web_data$Genotype)[r]
      %   subset=prior_web_data[which(prior_web_data$Genotype==gen),]
      %   for(c in 1:4){
      %     if(sum(subset[,2+c])>0){
      %       prior_web[r,c]=1
      %     } else {
      %       prior_web[r,c]=0
      %     }
      %   }        
      % }

      % Making prior galler-parasitoid web binary:
      gp_web_nonbinary=read.csv('../data/Salix_example/prior_web_para_only.csv',header=TRUE,sep=',',row.names=1)
      gp_web<-gp_web_nonbinary
      for(r in 1:nrow(gp_web)){
      for(c in 1:ncol(gp_web)){
      if(gp_web_nonbinary[r,c]>0){
      gp_web[r,c]<-1
      }
      }
      }

    \subsection*{Degree distributions and interaction probabilities}

      Obtaining degree distributions from the binary interaction 
      network is straightforward. To do this, we simply divide the 
      total number of observed interaction partners for each 
      species (row or column sums) with the number of potential 
      interaction partners (number of column or rows). Degree 
      distributions calculated, we can then calculate the 
      probability of each interaction as the product of the 
      normalized degrees of the two species. For example, the 
      probability of an interaction between two species with 
      normalized degrees of 0.5 (each interacts with half of the 
      available partners) is 0.25. R code used to do this follows:


      \vspace{12pt}
      \emph{
      \noindent \hspace{-4pt} \# Now get the degree distributions from the web\\
      deg\_dist\_Salix=rowSums(prior\_web)/ncol(prior\_web)\\
      \# Remove one genotype that never interacted\\
      deg\_dist\_Salix\textless-deg\_dist\_Salix[which(rowSums(prior\_web)\textgreater0)]\\
      deg\_dist\_galler=colSums(prior\_web)/nrow(prior\_web)\\
      \# Interaction probabilities are the product of plant and galler probabilities\\
      int\_probs=as.numeric(deg\_dist\_galler\%$*$\%t(deg\_dist\_Salix))\\
      }

      %       # Now get the degree distributions from the web
      % deg_dist_Salix=rowSums(prior_web)/ncol(prior_web)
      % # Remove one genotype that never interacted
      % deg_dist_Salix<-deg_dist_Salix[which(rowSums(prior_web)>0)]
      % deg_dist_galler=colSums(prior_web)/nrow(prior_web)
      % # Interaction probabilities are the product of plant and galler probabilities
      % int_probs=as.numeric(deg_dist_galler%*%t(deg_dist_Salix))

\clearpage


\section*{Appendix S2: Calculating posterior distributions and confidence intervals}

  The following simple functions are implemented in the R language. 
  Priordata is the list of interaction frequencies from the prior 
  data, n is the number of sites with observed co-occurrance of 
  species $i$ and $j$, and k is the number of sites with an 
  observed interaction $ij$. When calculating the prior 
  distribution, both n and k are 0. In the main text we assume 100\%
  confidence in observed interactions and therefore consider only 
  cases where k=0. Only n varies between species pairs.


  To calculate the parameters $\alpha$ and $\beta$ of a prior or 
  posterior distribution:

  \vspace{12pt}
  \emph{
  \noindent \hspace{-4pt}calculate\_parameters\textless-function(priordata,n,k)\{\\
  \hspace{4pt} \# Calculate prior parameters\\
  \hspace{4pt} start=list(shape1=1,shape2=1)
  \hspace{80pt} pars=fitdistr(x=priordata,`beta',start=start,lower=c(0,0))\$estimate\\
  \hspace{4pt} alpha=pars[[1]]\\
  \hspace{4pt} beta=pars[[2]]\\
  \hspace{4pt} \# Update the parameters with data. If n=0 and k=0, no change.\\
  \hspace{4pt} alpha\_prime=alpha+k\\
  \hspace{4pt} beta\_prime=beta+n-k\\
  \hspace{4pt} pars2=c(alpha\_prime,beta\_prime)\\
  \hspace{4pt} return(pars2) \}
  }
  \vspace{12pt}


    To calculate the maximum likelihood estimates of the mean and 
    variance of the probability of interaction $\lambda_{ij}$:

    \vspace{12pt}
    \emph{
    calculate\_distribution\textless-function(pars)\{\\
      \# alpha and beta may be from a prior or posterior distribution\\
      alpha=pars[[1]]\\
      beta=pars[[2]]\\


      \# Calculate the MLE of the mean\\
      mu\_num=alpha\\
      mu\_den=alpha+beta\\
      mu=mu\_num\/mu\_den\\


      \# Calculate the MLE of the variance\\
      sig\_num=alpha\*beta\\
      den1=alpha+beta\\
      den2=den1\*\*2\\
      sig\_den=den1\*den2\\
      sigma2=sig\_num\/sig\_den\\


      return(c(mu,sigma2))\}
    }
  \vspace{12pt}
      

  To calculate a credible interval based on the prior or posterior distribution, for given lower and upper bounds:


  \vspace{12pt}
  \emph{
  credible\_interval\textless-function(pars,p\_lower,p\_upper)\{\\
    alpha=pars[[1]]\\
    beta=pars[[2]]\\
    lowCI=qbeta(p=p\_lower,shape1=alpha,shape2=beta)\\
    highCI=qbeta(p=p\_upper,shape1=alpha,shape2=beta)\\
    return(c(lowCI,highCI))\}
  }


  \vspace{12pt}
  \emph{
  plot\_precision\textless-function(threshold,confidence,pars)\{\\
    alpha=pars[[1]]\\
    beta=pars[[2]]\\
    n=seq(0,100,1)\\
    k=0\\
    cdf=pbeta(threshold,shape1=alpha,shape2=beta+n)\\
    samples=length(which(cdf<confidence))\\
    return(samples) \}
  }


\clearpage

    \bibliographystyle{ecollett} 
    \bibliography{manual_abbrev} % Abbreviate journal titles.


\end{document}



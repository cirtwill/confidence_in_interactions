
\documentclass[12pt]{letter}

% \usepackage[britdate]{canterbury-letter}
\usepackage[britdate]{LiU-letter}
\usepackage{times}
\usepackage{letterbib}
\usepackage{geometry}
\usepackage[round]{natbib}
\usepackage{graphicx}
\geometry{a4paper}
\usepackage[T1]{fontenc}
\usepackage[utf8]{inputenc}
\usepackage{authblk}
\usepackage[running]{lineno}
\usepackage{amsmath,amsfonts,amssymb}
% \usepackage[margin=10pt,font=small,labelfont=bf]{caption}

%\usepackage{natbib}
% \bibpunct[; ]{(}{)}{;}{a}{,}{;}

\newenvironment{refquote}{\bigskip \begin{it}}{\end{it}\smallskip}

\newenvironment{figure}{}

\position{Postdoctoral fellow}
\department{Department of Physics, Chemistry, and Biology (IFM)}
\location{581 83 Link\"{o}ping, Sweden}
\cityzip{581 83 Link\"{o}ping, Sweden}
\telephone{}
\fax{}
\email{alyssa.cirtwill@liu.se}
\url{http://cirtwill.github.io}
\name{Alyssa R. Cirtwill}

\newcommand{\myjournal}{\emph{Ecology Letters}}

\begin{document}

\begin{letter}{\bf Professor Tim Coulson\\
               Editor-in-Chief, Ecology Letters\\
               Department of Zoology\\
               University of Oxford\\
               Oxford OX1 3PS, UK
                               }


\opening{Dear Prof. Coulson:}

    My co-authors and I cordially invite you to consider the 
    following proposal for an Ideas \& Perspectives article in 
    \myjournal:



Researchers working with ecological networks widely acknowledge that empirical data come with major caveats: the set of observed interactions included in such networks depends on the local conditions and sampling methodology used as well as on the set of interactions that actually occur. An interaction may not be observed because 1) it does not occur, 2) species which could interact are too rare to encounter each other, inclement weather or other conditions prevented the interaction, etc., or 3) detection of interactions was not perfect. From the perspective of an interaction matrix, these sources of non-interaction are indistinguishable. We propose a Bayesian framework with which to quantify the uncertainty around both particular interactions and network-level metrics. We will illustrate this framework using an extraordinarily large and well-replicated set of plant-galler-parasitoid networks. By quantifying the uncertainty around interactions, network ecologists can more rigorously compare networks, identify patterns which should be more robust to variation in sampling, develop better models and predictions, etc.

The authors involved in this study bring varied perspectives to this discussion. Tomas Roslin has extensive expertise in gathering network data and has first-hand experience with each source of variation we describe. Dominique Gravel and Anna Ekl\"{o}f both offer a more theoretical perspective and have extensive experience with probabilistic approaches to network analyses. The two more junior authors, Kate Wootton and Alyssa Cirtwill, are involved with a variety of network analyses using both empirical and simulated networks and have a strong interest in developing more robust methods for network analyses.

We are confident that the applicability of this method to all types of ecological networks and the importance of addressing sources of uncertainty in the data will mean that this article is of interest to a broad readership. We hope that you agree and eagerly await your reply.



\closing{Best regards,}


\end{letter}
\end{document}

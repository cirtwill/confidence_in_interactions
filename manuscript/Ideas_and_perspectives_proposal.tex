
\documentclass[12pt]{letter}

% \usepackage[britdate]{canterbury-letter}
\usepackage[britdate]{LiU-letter}
\usepackage{times}
\usepackage{letterbib}
\usepackage{geometry}
\usepackage[round]{natbib}
\usepackage{graphicx}
\geometry{a4paper}
\usepackage[T1]{fontenc}
\usepackage[utf8]{inputenc}
\usepackage{authblk}
\usepackage[running]{lineno}
\usepackage{amsmath,amsfonts,amssymb}
% \usepackage[margin=10pt,font=small,labelfont=bf]{caption}

%\usepackage{natbib}
% \bibpunct[; ]{(}{)}{;}{a}{,}{;}

\newenvironment{refquote}{\bigskip \begin{it}}{\end{it}\smallskip}

\newenvironment{figure}{}

\position{Postdoctoral fellow}
\department{Department of Physics, Chemistry, and Biology (IFM)}
\location{581 83 Link\"{o}ping, Sweden}
\cityzip{581 83 Link\"{o}ping, Sweden}
\telephone{}
\fax{}
\email{alyssa.cirtwill@liu.se}
\url{http://cirtwill.github.io}
\name{Alyssa R. Cirtwill}

\newcommand{\myjournal}{\emph{Ecology Letters}}

\begin{document}

\begin{letter}{\bf Professor Tim Coulson\\
               Editor-in-Chief, Ecology Letters\\
               Department of Zoology\\
               University of Oxford\\
               Oxford OX1 3PS, UK
                               }

% The instructions are for a single, 300-word paragraph proposal. The large paragraph is now at 299 words, and we can maybe hope that they won't count the preamble and nice closing.

\opening{Dear Prof. Coulson:}

    My co-authors and I cordially invite you to consider the 
    following proposal for an Ideas \& Perspectives article in 
    \myjournal:


    Representing communities as networks of interactions (links) between taxa (nodes) is a central approach in modern community ecology. These networks are usually treated as fixed, deterministic entities. Yet, researchers working with ecological networks widely acknowledge that empirical data are in fact samples of a community and come with major caveats. The probability of observing different interactions depends on sampling effort/methodology and local conditions as well as the intrinsic probability of any interaction. These contingencies mean that there are many reasons one might fail to observe an interaction (e.g., it might occur sometimes but not during sampling due to weather). Problematically, these sources of non-interaction are often indistinguishable within an interaction matrix. As a potential solution, we propose a Bayesian framework which partitions the uncertainty associated with missing interactions into explicit components and allows us to place confidence intervals around each estimated interaction probability. The Bayesian framework also facilitates the use of prior information (e.g., species traits) to reduce uncertainty around some interactions. We will illustrate this framework using an extraordinarily  well-replicated set of plant-galler-parasitoid networks recently published as a data paper (Kopelke \emph{et al.}, 2017. \textbf{Ecology} 98: 1730). Using this dataset, we will demonstrate how uncertainty surrounding each interaction probability affects network-level metrics. Acknowledging this uncertainty is crucial for valid comparison of networks along gradients. The authors involved in this study contribute a range of valuable perspectives. Dominique Gravel has developed techniques for novel types of interaction distribution modelling. Anna Ekl\"{o}f contributes extensive experience with probabilistic network analyses. Tomas Roslin has vast experience with gathering and using network data and has first-hand experience with each source of variation we describe. The two junior authors, Kate Wootton and Alyssa Cirtwill, work with both empirical and simulated networks and have a strong interest in developing more robust methods for network analyses.

    
    We are confident that the applicability of this method to all types of ecological networks and the importance of addressing sources of uncertainty in the data will mean that this article is of interest to a broad readership. We hope that you agree and eagerly await your reply.



\closing{Best regards,}


\end{letter}
\end{document}

\documentclass[12pt]{article} 
\usepackage{amsmath} 
\usepackage[dvips]{graphicx}
\usepackage{multirow} 
\usepackage{geometry} 
\usepackage{pdflscape}
\usepackage[labelfont=bf]{caption} 
\usepackage{setspace}
\usepackage[running]{lineno} 
% \usepackage[numbers,sort]{natbib}
\usepackage[round]{natbib} 
\usepackage{array}

\newcommand{\methods}{\textit{Materials \& Methods}}
\newcommand{\SI}{\textit{Appendix}~}

\topmargin -1.5cm % 0.0cm 
\oddsidemargin 0.0cm % 0.2cm 
\textwidth 6.5in
\textheight 9.0in % 21cm
\footskip 1.0cm % 1.0cm

\usepackage{authblk}

\title{???A quantitative framework for investigating the reliability of network construction}


\author{Dominique Gravel, D. Carstensen, Timoth\'{e}e Poisot, Daniel B. Stouffer, \& Alyssa R. Cirtwill$^{1}$, others??}
\date{\small$^1$Department of Physics, Chemistry\\ 
and Biology (IFM)\\ 
Link\"{o}ping University\\
Link\"{o}ping, Sweden\\
% \medskip
% $^\dagger$ Corresponding author:\\
% alyssa.cirtwill@gmail.com\\
% +46 723 158464\\
 }

\renewcommand\Authands{ and }

\begin{document} 
\maketitle 
\raggedright
\setlength{\parindent}{15pt} 


\section*{Abstract}

% A quantitative framework for investigating the reliability of network construction
% D. Gravel, D. Carstensen, T. Poisot, D. Stouffer, A. Cirtwill

\section*{Introduction}

    % Present Salix data with holes as an example of uncertainty in data. Focus just on the white and grey, ignore uncertainty in the blacks


    Ecological networks-- which include antagonistic food webs and host-parasite networks as well as plant-pollinator and plant-frugivore mutualist networks --are usually considered to be static representations of the communities and interactions they describe. That is, whether the network is assembled based on aggregated data, a single intensive "snapshot" sample, or expert knowledge, interactions are assumed to occur at constant frequencies~\citep{Olesen2011a}. This assumption conflicts with known variation in interactions over time~\citep{Kitching1987,Olesen2011a} and space~\citep{Kitching1987,Baiser2012} and between individuals of a given species~\citep{Pires2011a,Fodrie2015,Novak2015}. Taking the variances in interaction frequencies into account would therefore represent a substantial increase in the realism of ecological networks as representations of ecological communities.


    Efforts to account for variance in interaction frequencies are complicated by the fact that not only interaction frequencies but network structure vary over space and time. Community composition and species abundances vary from site to site~\citep{Baiser2012} and over time within a site~\citep{Olesen2011a}. Moreover, interactions between species may change even if the participating species remain present at a site but do not co-occur temporally or are too rare to detect each other~\citep{Tylianakis2010}, or through changes to individual preferences~\citep{Fodrie2015}. Several researchers have pointed to the importance of sampling intensity for the assessment of network structure (e.g.,~\citealp{}). The acknowledgement that insufficient sampling effort limits our ability to describe variation in ecological networks has not, however, led to a quantitative framework to deal with the uncertainty of ecological interactions and of sampling.


    The main objective of this study is to develop analytical tools that could be used to better represent the uncertainty in the estimation of pairwise interaction probability. We start with the description of a new Bayesian approach to estimate ecological interactions and of a related R package. We first illustrate the framework with simple quantitative examples, followed by an assessment of the quality of an empirically-estimated plant-pollinator network. Next, we use the framework to provide analytical criteria for improving the sampling of rare interations. Finally, we discuss the implications of uncertainty in the estimation of pairwise interactions on the properties of ecological networks. Through these efforts, we demonstrate both the utility of our approach and the importance of acknowledging the uncertainty inherent in ecological networks.


% <!-- ## Background

%   - Ecological interaction networks are usually considered deterministic, but nature is highly variable and interactions should be considered with their variance
%   - There is spatial and temporal variability in network structure and of pairwise interactions
%   - There used to be a debate on the importance of sampling intensity for the assessment of network structure, but it never led to a quantitative framework dealing with uncertainty of ecological interactions and of sampling

% ## Objective:

%   - The main objective of this study is to develop analytical tools that could be used to better represent the uncertainty in the estimation of pairwise interaction probability
%   - We start with the description of a new bayesian approach to estimate ecological interactions and of a related R package. 
%   - We illustre the framework with simple quantitative examples, and then with the assessment of the quality of an empirical estimate of plant-pollinator networks. 
%   - We further use the framework to provide analytical criteria for improving the sampling of rare interations. 
%   - We discuss the implications of the uncertainty in the estimation of pairwise interactions on the properties of ecological networks -->

\section*{Pairwise interactions as a Bernoulli process}

    We consider the occurrence of an ecological interaction as a stochastic process. Provided that the two species could interact, this interaction could happen with a given probability. This probability is likely to be less than one (i.e., the interaction may not occur) because i) the two species may not encounter each other during a given time interval and within a given area, ii) there may be unknown environmental drivers affecting the realization of the interaction or iii) a third species might interfere and prevent the interaction from happenning (Poisot2015). 


    We consider that the occurrence of an interaction is a Bernoulli trial. We define $\theta$ as the probability that two species interact with each other. Consequently, the number of success $X = k$ over $n$ trials will follow a binomial distribution: 
    
    \begin{equation}
      X \sim Bin(n,\theta) ,
    \end{equation}

    \noindent and 

    \begin{equation}
       P(X = k|\theta,n) = {n \choose k}\theta^k(1-\theta)^{n-k} . 
       \label{likelihood}
    \end{equation}


    \noindent The parameter $\theta$, the probability of observing an interaction over a given time interval
    and area, is the quantity we want to estimate from empirical data. The variance of a 
    Bernoulli experiment is simply n$\theta$(1-$\theta$).


    The maximal likelihood estimate of $\theta$ is straightforward to find given $X=k$ and $n$. It is simply equal to:


    \begin{equation}
      \theta_{MLE} = \frac{k}{n}  .
      \label{theta_MLE}
    \end{equation}


    It is possible to compute the confidence interval for this estimate using any of several methods, including the \emph{Wilson score interval}, the \emph{Clopper-Pearson interval}, and the \emph{Agresti-Coull Interval}. Finding this estimate is therefore quite straightforward, but it nonetheless has two drawbacks. First, $\theta$ is a random variable with a given uncertainty and whose distribution is unknown, it is not a single point estimate. Second, the number of trials $n$ might be very low in many instances (and some pairs of species might even not be documented), and consequently the uncertainty in the estimation of $\theta$ might be considerable. We therefore propose a Bayesian approach to solve these two problems. 


    % Add section explaining process variation and detection probability here, how that will affect our ability to estimate.


\section{Why some interactions do not occur ?}

It is getting more and more common to represent interactions as probabilities. Some theoretical models of network structure propose to represent interactions with probabilities, for instance the minmum potential model introduces gaps in the feeding niche of predators (Allesina et al. 2008), while a gaussian probability function is used to represent the niche by Williams et al. (2010). Eklof et al. (2013) introduced variability as well in the computation of network dimensions, allowing the absence of interactions within the n-dimension niche of every species. Empirical investigation of trait-matching constraints often consider the occurrence of interactions as binomial process (e.g. Rohr et al. 2016). There are also ways to perform analysis of probabilistic networks, with almost all metrics having a probabilistic version (Poisot et al. 2015). But there is not yet consensus on what an interaction probability between a pair of species means, and this is definitely a problem when comes time to estimate the uncertainty of both pairwise interactions and entire networks. 

Starting from the problem described in the previous section, we consider that a pair of species have been observed $n$ times co-occurring at one location, and the number of times they interact is $k = 0$. We aim to evaluate the uncertainty of this interaction. What this is mean exactly ? We consider there are three nested level of uncertainty making this phenomenon a stochastic process :  

\textit{Interaction uncertainty} First, and most fundamentally, we do not know if these species have the appropriate characteristics to interact or not. We define the probability of an interaction $L$ as $P(L)=\lambda$. Obviously, because $k=0$, it is very likely they can not interact if provided enough time and there were no environmental constraint for this interaction to happen. But it is nonetheless possible the interaction is a rare phenomenon that has never been documented before. This source of uncertainty is the one documented by trait-matching models. It arises because every model is imperfect and lack information (i.e. traits) about the constraints on the interaction. In other words, with sufficient sampling and all information accessible, this interaction probability should either tend to 0 or to 1 and the uncertainty vanish.  

\textit{Process uncertainty} It could happen that the interaction is feasible, i.e. $L=1$, but that it does not occur at a given location or at a moment in time because there are local constraints preventing it to occur. We define the realization of the interaction process $X$ given an interaction is feasible as a stochastic process with associated probability $P(X|L)=\chi$. This phenomenon of interaction contingencies is usually not considered in network studies, but there is a rich literature in community ecology about the contingencies of interactions. We imagine for instance that an interaction between a gall and a parasitoid is not recorded at a location because it was the wrong time of the year and the gall has not yet formed (phenological constraints), because a late frost killed the larvae (an abiotic environmental constraint) , another parasitoid species competitively excluded the parasitoid species of interst (a biotic environmental constraint) or simply the parasitoid is too rare (an abundance constraint). 

\textit{Detection uncertainty} Lastly, measurement errors are always source of uncertainty in the observation of ecological processes. We define the detection of an interaction $D$, given an interaction is feasible and occurs in the local conditions, as a stochastic process with associated probability $P(D|X,L)=\delta$. Detection failure could happen for several reasons, for instance if the rearing of a parasitoid fails with inappropriate lab conditions or because of species mis-identification. Some source of detection error could be minimized with appropriate sampling effort ($\chi$ will converge to one with increasing number of collected galls), but other sources are often difficult to reduce (e.g. the occurrence of cryptic species might require molecular analysis for appropriate taxonomic identification).

\textit{What is an observation of "no interaction" ?} The combination of these three sources of uncertainty together leads to many potential explanations for the observation of an absence of interaction, but only the situation where $L = 0$ is a true absence therefore is relevant for the investigation of networks. While the observation of an interaction is straightforward to interpret, absence of interactions must be decomposed in different quantities. It is particularly important to rule out the situations where $D=0 \cup X = 1 \cup L=1$, i.e. the interaction occurred at the location but was not observed, and $D=1 \cup X = 0 \cup L =1$, the interaction would have been detected but did not occur. 

It is important to keep in mind that for network analysis, ecologists seek to measure the occurrence of true absences, which is the joint event $L=1\cup X=1 \cup \D=1$), but in reality they measure the marginal probability $P(L) = k/n$. This reasonning explains why the above describe MLE estimation of the interaction probability is naive, and could be inappropriate in many instances, in particular when we have no knowledge of the detection and the process uncertainties.   


  \section*{Bayesian approach to infer interaction probabilities}

    \subsection*{Posterior distribution of the interaction probability}

  We adopt a bayesian approach to estimate the distribution of the parameter $\theta$. According to Bayes principle, the posterior distribution of $\theta$ is:

  \begin{equation}
    \underbrace{P(\theta|X,n)}_{Posterior} = \frac{\overbrace{P(X|\theta,n)}^{Likelihood}\overbrace{P(\theta)}^{Prior}}{\underbrace{P(X|N)}_{Normalizer}} .
    \label{posterior}
  \end{equation}

  According to the above description, the likelihood is simply the binomial distribution (Eq.~\ref{likelihood}). Since $\theta$ is a probability it is bounded between 0 and 1 and consequently the most appropriate prior distribution is the beta:

    \begin{equation}
      \theta \sim Beta(\alpha,\beta) , \label{prior}
    \end{equation}


  \noindent which has two shape parameters, $\alpha$ and $\beta$. 


  It might be complicated in many cases to compute the normalizer, but fortunately there is an analytical solution to the normalizer. The beta-binomial distribution is a conjugate distribution of the binomial distribution. This allows us to analytically compute the posterior distribution of a binomial model with a beta prior distribution. We can re-write the posterior distribution of $\theta$ as:


  \begin{equation}
    P(\theta|k,n) = \frac{\theta^{\alpha+k-1}(1-\theta)^{\beta+n-k-1}}{B(\alpha+k,\beta+n-k)} , \label{posterior}
  \end{equation}

  \noindent where the function $B$ is the beta function:

  \begin{equation}
    Beta(\alpha+k,\beta+n-k) = \frac{\Gamma(\alpha+k)\Gamma(\beta+n-k)}{\Gamma(\alpha+\beta+n)} . \label{betafunction}
  \end{equation}

  The posterior distribution of $\theta$ therefore follows the beta distribution with new parameters $\alpha'= \alpha+k$ and $\beta'=\beta+n-k$. The weight of the prior on the posterior distribution is understood from these definition of the parameters: the difference between the posterior and the prior will increase with $k$ and $n-k$. When plotted, we find the shape of the distribution gets narrower with $k$ and $n$ (see Fig X). 


  The posterior distribution of $\theta$ could be computed with R with the following command:


  \vspace{12pt}
  \noindent\emph{dbeta(x = theta, shape1 = alpha+k, shape1 = beta+n-k)}
  \vspace{12pt}


  \subsection*{Moments and other properties}

  The fact that the posterior distribution of $\theta$ follows a beta distribution makes it straightforward to compute moments and other properties. 

  The \textbf{average} of $\theta$ is: 
      \begin{equation}
        \bar{\theta} = \frac{\alpha+k}{\alpha+\beta+n} ,
        \label{mean}
      \end{equation}


    and its \textbf{variance} is:  
      \begin{equation}
        Var(\theta|k) = \frac{(\alpha + k)(\beta + n - k)}{(\alpha + \beta + n)^{2}(\alpha + \beta + n +1)}
        \label{variance}
      \end{equation}


    The \textbf{mode} of the distribution (which is not necessarily the maximum likelihood estimate) is:
      \begin{equation}
        \hat{\theta} = \frac{\alpha + k - 1}{\alpha + \beta + n - 2} .
        \label{mode}
      \end{equation}


  \subsection*{The prior distribution}
    
  Parameters $\alpha$ and $\beta$ determine the shape of the prior distribution, which follows a beta distribution. These are call hyper parameters. We identify at least four ways to formulate the prior distribution of $\theta$. 


    \subsubsection*{Uninformative prior}
      
      In absence of any external information, an uniformative prior is the most conservative 
      hypothesis for the distribution of $\theta$. The beta distribution is uninformative and follows 
      a uniform distribution for $Beta(\alpha=1,\beta=1)$. 


    \subsubsection*{Distribution of connectance}
      
      There is now a great collection of ecological network data for which know the distribution of connectance. Connectance of a network is measured as $C = L/S^2$, where $L$ is the number of links and $S$ is the number of species. It measures the filling of an interaction matrix and also expresses the average probability that any two species interact with each other. If one  knows only the mean $\overline{C}$ and the variance $\sigma_C^2$ of the distribution of $C$, then the beta parameters could be computed as follows using the method of moments:

  \begin{equation}
    \alpha = \overline{C}(\frac{\overline{C}(1-\overline{C})}{\sigma_C^2}-1) ,
  \end{equation}

  \begin{equation}
    \beta = (1-\overline{C})(\frac{\overline{C}(1-\overline{C})}{\sigma_C^2}-1) .
  \end{equation}
  
  It is also possible to compute maximum likelihood estimates for a sample of the distribution of $C$. The following piece of R code provides an example: 

  \vspace{12pt}
  \noindent\emph{
    library(MASS)\\
    pars = fitdistr(x = vecC, "beta", start = list(shape1 = 1, shape2 = 1))\$estimate
  }
  \vspace{12pt}


  Where $vecC$ is a vector of known connectances for a set of networks. 


  \subsubsection*{Degree distribution}. 

  Connectance is an average property for a given network, it represents the expected degree standardized by the number of species in the network. The standardized degree could therefore be interpreted as an interaction probability. It is consequently possible to use the degree distribution to fuel the prior distribution. The degree distribution could come from several networks, or from the network of interest if interaction probabilities for some species are already documented. The latter approach allows us to apply information from known, abundant species to the rarest species for which interactions are less frequently documented. The procedure for the estimation of the hyper parameters follows exactly the same approach as described above for connectance except that each measurement is at the species level instead of the network level. 


  \subsubsection*{Trait-matching function}. 

  It is possible to estimate the probability of interaction between a pair of species if there is knowledge of their traits and some functions relating trait matching to interaction probability (Morales-Castilla2015). There are several techniques published to perform this inferrence of interaction probability (REFS). Note that in this case the prior might not be beta distribution and numerical methods might be required to compute the posterior distribution.  


\section*{A quantitative example}

The framework is best illustrated with a simple quantiative example. Suppose we have $n = 10$ observations of co-occurrence between speices $i$ and species $j$ in a given time interval and area, and $X = 2$ observations of interactions. The maximum likelihood estimate of the interaction probability is simply $\theta_{MLE} = 2/10 = 0.2$.  Now consider we know that species $i$ is known to interact with 10 other species, which have the following standardized degrees:

\vspace{12pt}
\noindent\emph{
    degree = c(0.66, 0.09, 0.06, 0.08, 0.04, 0.30, 0.06, 0.38, 0.03, 0.19)
    }.
  \vspace{12pt}


Using these standardized degrees to inform our prior expectation of the 
distribution of interaction probabilities and the R code from the 
connectance example, we obtain prior parameters $\alpha$=0.9034876 
and $\beta$=3.6623995. Using these priors in equations~\ref{mean} 
and~\ref{variance} above, we find an average interaction probability of 0.1993348 and a variance of 0.01025322 [[Different from Dom's values]].
% n=number of observed co-occurances.
The average interaction probability is 0.2163971 and the variance is 0.02353319. The hyper parameters are computed according to the following script, using the function \emph{fitdistr} from the R~\citep{R} package MASS~\citep{MASS}, yielding:

\vspace{12pt}
\noindent\emph{    pars = fitdistr(x = degree, "beta", start = list(shape1 = 1, shape2 = 1),lower=c(0,0))\$estimate }.
  \vspace{12pt}


The lower bound of (0,0) for the two parameters of the Beta distribution prevents the function from attempting to fit invalid (negative) parameter estimations.

The average of the distribution is computed as follows:

  \vspace{12pt}
\noindent
\emph{
    \noindent n = 10
    \\\noindent k = 2
    \\\noindent alpha\_prime = alpha + k
    \\\noindent beta\_prime = beta + n - k
    \\\noindent mean\_theta = alpha\_prime/(alpha\_prime+beta\_prime)
},
  \vspace{12pt}


which yields a value $E[\theta]=0.2057651$ [[I get 0.1993348]]. And the variance:

  \vspace{12pt}
\noindent\emph{alpha\_prime$\times$ beta\_prime\/(alpha\_prime+beta\_prime)$^2$\/(alpha\_prime+beta\_prime+1)},
  \vspace{12pt}


which yiealds a value $var[\theta] = 0.009579623$. [[I get 0.01025322]]


The credible interval is not symmetric around the mean since the distribution is bounded between 0 and 1 and is not symmetric. In this case, a 95\% credible interval for the mean of theta is (0.0448479, 0.4308686) and can be computed using the quantile function \emph{qbeta}:

  \vspace{12pt}
\noindent\emph{
    \noindent lowCI = qbeta(p = 0.025, shape1 = alpha\_prime, shape2 = beta\_prime)
    \\\noindent highCI = qbeta(p = 0.975, shape1 = alpha\_prime, shape2 = beta\_prime)}.
  \vspace{12pt}


Now, consider the case where the two species have never been observed interacting across $n$ trials. The question is then "what is the probability these two species do not interact"? Since it will never be possible to prove that the two species could never interact, given the framework above we must fix a threshold below which we consider that there is no interaction. Let this threshold probability be called $\theta*$. We use the cumulative distribution function to estimate $P(\theta<\theta*|X=0,n)$. The following script illustrates this probability for an increasing number of trials and $\theta*$=0.1:

  \vspace{12pt}
\noindent\emph{
    \noindent n = seq(0,100,1)
    \\\noindent X = 0
    \\\noindent theta\_star = 0.1
    \\\noindent cdf = pbeta(theta\_star, shape1 = alpha+X, shape2 = beta+n-X)
    \\\noindent plot(n,cdf, type = "l", xlab = "n", ylab = "Cumulative distribution")
    \\\noindent abline(h = 0.95,lty = 3)}.
  \vspace{12pt}


These trials yield a surprising result: it requires \textgreater30 [[or 23??]] observations of no interactions to be 95\% sure that the interaction probability is smaller than 0.1. Note the special case where there is no observation of the two species co-occurring and failing to interact, $n = 0$. In this situation, the posterior distribution converges to the prior distribution. 


\section*{Case study: estimating the uncertainty of a plant-herbivore network}

    \subsection*{Description of the data}

      \emph{Salix}-galler-parasitoid meta-network, here focusing on the smaller \emph{Salix}-galler network. The meta-network consists of interactions between 52? \emph{Salix} species and 222 species of gall-forming insects, sampled from 374 locations in Europe ranging from Sicily to the Arctic. In total, 4295 unique interactions were observed. Each of the 374 locations can be considered as a network in its own right, but here we consider each network as an independent sample from which to build the meta-network.


  \subsection*{Finding the maximum likelihood estimate}

      In a strict Bayesian framework, we wish to use a prior distribution that does not rely on any information from the study at hand. To that end, we use data from another well-described \emph{Salix} galler-parasitoid system (doi: 10.1073/pnas.1513633113). Note that this study used several genotypes of \emph{S. hookeriana} rather than different \emph{Salix} species. We take the degree distributions of four Cecidomyid galler species across \emph{S. hookeriana} genotypes as our prior distribution for the probability of interactions between gallers and \emph{Salix} species. Note that as an interaction frequency depends on the product of the degree of the galler and the degree of the plant involved, we wish to use as our prior the product of the degree distributions of plants and gallers.
      Unfortunately, the product of two beta distributions is not beta-distributed. Instead, we will use the list of interaction probabilities, obtained by multiplying the two degree distributions.


      The degree distribution across \emph{S. hookeriana} genotypes was calculated from data deposited in Dryad using the following R code:

      prior_web=read.csv(path_to_Dryad_file)
      deg_dist_Salix=with(prior_web,tapply(gall_total_rich,Genotype,mean))/4
      gallers=cbind(prior_web$Genotype,prior_web[,27:30])
      aSG_means=with(prior_web,tapply(aSG_abund,Genotype,mean))
      rG_means=with(prior_web,tapply(rG_abund,Genotype,mean))
      vLG_means=with(prior_web,tapply(vLG_abund,Genotype,mean))
      SG_means=with(prior_web,tapply(SG_abund,Genotype,mean))
      deg_dist_galler=c(length(which(aSG_means>0)),length(which(rG_means>0)),length(which(SG_means>0)),length(which(rG_means>0))      )/26
      int_probs=as.numeric(deg_dist_galler\%*\%t(deg_dist_Salix))

      pars=fitdistr(x=int_probs,"beta",start=list(shape1=1,shape2=1))$estimate
      alpha=pars[[1]]
      beta=pars[[2]]
      mu=alpha/(alpha+beta)
      num=alpha*beta
      den=(alpha+beta)**2*(alpha+beta)
      sigma2=num/den


      Using this degree distribution we can obtain prior values for $\alpha$ and $\beta$ using the code above. We obtain
      $\alpha$=2.175656, $\beta$=5.361083.

      For species where no co-occurances were observed, we can calculate the MLE estimates for the mean and variance of $\omega_{ij}$ directly from this prior. These are:
      $\hat{\omega_{ij}}$=0.2886734, var($\omega$)=0.02724535.


      For a pair of species where some co-occurances were observed, we base these estimates on the prior distribution above conditioned by the available data. If we consider only pairs of species which were observed to co-occur but not to interact, $k_{ij}$ is always 0 and only $n_{ij}$ will vary between species pairs. Thus $\alpha'$=$\alpha$ and $\beta'$=$\beta + n$. At the most extreme case, for a pair of species which co-occurred at all 374 sites and was never observed to interact, our distribution would become:

      $\hat{\omega_{ij}}$=0.005702349, var($\omega$)=1.486051e-05.
      

      Realistic values for n when k=0, based on a sample of 20 random sites:
      n=1: 0.25485794 0.02224566
      n=2: 0.22813413 0.01846427
      n=3:
      n=4:






    \subsection*{Computing the posterior distribution}

  \subsection*{Computing the confidence interval around a probability estimate}

    \subsection*{How many samples are required to reach a minimal precision}


\clearpage

    \bibliographystyle{ecollett} 
    \bibliography{manual_abbrev} % Abbreviate journal titles.


\end{document}



\documentclass[12pt]{letter}

% \usepackage[britdate]{canterbury-letter}
\usepackage[britdate]{LiU-letter}
\usepackage{times}
\usepackage{letterbib}
\usepackage{geometry}
\usepackage[round]{natbib}
\usepackage{graphicx}
\geometry{a4paper}
\usepackage[T1]{fontenc}
\usepackage[utf8]{inputenc}
\usepackage{authblk}
\usepackage[running]{lineno}
\usepackage{amsmath,amsfonts,amssymb}
% \usepackage[margin=10pt,font=small,labelfont=bf]{caption}

%\usepackage{natbib}
% \bibpunct[; ]{(}{)}{;}{a}{,}{;}

\newenvironment{refquote}{\bigskip \begin{it}}{\end{it}\smallskip}

\newenvironment{figure}{}

\position{Postdoctoral fellow}
\department{Department of Physics, Chemistry, and Biology (IFM)}
\location{581 83 Link\"{o}ping, Sweden}
\cityzip{581 83 Link\"{o}ping, Sweden}
\telephone{}
\fax{}
\email{alyssa.cirtwill@liu.se}
\url{http://cirtwill.github.io}
\name{Dr. Alyssa R. Cirtwill}

\newcommand{\myjournal}{\emph{Ecology Letters}}

\begin{document}

\begin{letter}{\bf Professor Rob Freckleton\\
               Executive Editor, Methods in Ecology and Evolution\\
               Department of Animal and Plant Sciences\\
               University of Sheffield\\
               Sheffield S10 2TN, UK
                               }

% Borrowing heavily from the proposal

\opening{Dear Prof. Freckleton:}

    My co-authors and I are pleased to submit the attached manuscript \emph{A quantitative framework for investigating the reliability of empirical network construction} for consideration as a research article in \textbf{Methods in Ecology and Evolution}. This manuscript is aimed squarely at the audiencce of \textbf{Methods in Ecology and Evolution}: ecologists in need of new tools for addressing topical questions in ecology and for recognizing key limitations and caveats in their research. Thus, the goals of this manuscript are to demonstrate the extreme difficulty of observing all interactions that occur within a community and, given this difficulty, to provide a way forward: a framework for combining prior knowledge about the system with observed data in order to gauge the probability of interactions which were not observed. We close by offering concrete recommendations that will facilitate researchers' efforts to acknowledge and set bounds on the uncertainty inherent in ecological data.


    % We argue that, because of multiple levels of uncertainty surrounding interactions between species, the set of observed interactions will almost certainly be a subset of the interactions that can occur in a community. Interactions may not be observed because a species is rare and does not encounter its partner during sampling, because environmental constraints prohibit the interaction during sampling, because one or both species is difficult to observe in the field, or because the interaction truly does not occur. We emphasise that some of these sources of uncertainty can be reduced with increased sampling but that, because it is easier to observe species than interactions between species, increased sampling will tend to add more unobserved than observed interactions to a matrix. We illustrate the problem using an extensively-sampled, spatially-replicated \emph{Salix}-galler-natural enemy system (Kopelke \emph{et al.}, 2017. \textbf{Ecology} 98: 1730).


    % In many cases, it will not be feasible to sample all species pairs intensively enough (tens of observations per pair) to be confident that the pair does not interact. Given the time and resource constraints that often restrict fieldwork, we suggest that researchers supplement observed data with prior information about the system. Thus, they can set bounds on the probability of interaction between species that were not observed interacting. We demonstrate this simple Bayesian framework, including the selection of a prior distribution, and use it to predict the likely set of feasible interactions in our example \emph{Salix} system. Based on this framework, we close with a set of recommendations for both empiricists and theoreticians that should improve our ability to deal with the uncertainty inherent in ecological networks and compare networks across systems.


    As our central approach to this topic, we use a simple Bayesian framework. We are aware that there are several studies already using various forms of Bayesian networks to treat interactions as probabilities. These efforts are, however, generally confined to fairly theoretical work. We believe that empirical ecologists will find Bayesian interaction probabilities to be a useful tool when planning field sampling and when describing and analysing their focal communities. Our goal with this manuscript is, therefore, to introduce a Bayesian line of thought in as approachable and broadly-applicable terms as possible. To that end, we provide simple R code that illustrates each step of our approach. 


    We have uploaded an earlier version of this manuscript to biorXiv \\(doi: \emph{https://doi.org/10.1101/332536}). The present version has been substantially re-organized and contains additional material not included in the pre-print.


    We hope that the general nature of uncertainty around interactions, as well as the broadly-applicable solution we offer, will be of interest to the readership of \textbf{Methods in Ecology and Evolution}. We are confident that our worked example and easy-to-use scripts will provide a useful template for empiricists and theoreticians alike. We hope that you agree and eagerly await your reply.


\closing{Best regards,}


\end{letter}
\end{document}


% Suggested reviewers:

% Pedro Jordano pedroj@me.com Estacion Biologica de Donana CSIC

% Phillip Staniczenko pstaniczenko@sesync.org National Socio-Environmental Synthesis Center (SESYNC) 

% Louis-Félix Bersier louis-felix.bersier@unifr.ch Université de Fribourg - co-author with DG in 2016

% Paulo Guimaraes prguima@gmail.com Universidade de Sao Paola

% Jason Tylianakis jason.tylianakis@canterbury.ac.nz :) - co-author with DG in 2016

% Owen Petchey University of Zurich owen.petchey@ieu.uzh.ch

% Eve McDonald-Madden e.mcdonaldmadden@uq.edu.au University of Queensland

% Laura Burkle laura.burkle@montana.edu Montana State University Dept. of Ecology - co-author with DG in 2017

% Jane Memmott jane.memmott@bristol.ac.uk University of Briston, School of Biological Sciences

% Sonia Kefi sonia.kefi@umontpellier.fr Institut des Sciences de l'Evolution de Montpellier co-author with DG in 2017, 2016; co-author with AE in 2017

% Charles Godfray charles.godfray@zoo.ok.ac.uk Oxford, Dept. of Zoology

% Catherine Graham catherine.graham@wsl.ch Swiss Federal Research Institute (WSL) - co-ahtour with DG in 2018, TR in 2018

% Extra suggestions
% Jacopo Grilli jgrilli@santafe.edu Santa Fe Institute
% Ben Weinstein Oregon State University weinsteb@oregonstate.edu
% Jean-Philippe Lessard Concordia University jp.lessard@concordia.ca
% Catherine Graham catherine.graham@wsl.ch Swiss Federal Research Institute WSL 
% Anne Chao chao@stat.nthu.edu.tw National Tsing Hua University
% Robert Colwell robertkcolwell@gmail.com Museum of Natural History, University of Colorado Boulder
% Mari Moora mari.moora@ut.ee (as working on fungal interaction networks, but being well on the map)
% (Anna Traveset atraveset@uib.es )
% Ignasi Bartomeus nacho.bartomeus@gmail.com 
% Owen Lewis owen.lewis@zoo.ox.ac.uk




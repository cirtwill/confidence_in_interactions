\documentclass[12pt]{letter}

\usepackage[britdate]{LiU-letter}
\usepackage{times}
\usepackage{letterbib}
\usepackage{geometry}
\usepackage[round]{natbib}
\usepackage{graphicx}
\geometry{a4paper}
\usepackage[T1]{fontenc}
\usepackage[utf8]{inputenc}
\usepackage{authblk}
\usepackage[running]{lineno}
\usepackage{amsmath,amsfonts,amssymb}
\usepackage[margin=10pt,font=small,labelfont=bf]{caption}

%\usepackage{natbib}
% \bibpunct[; ]{(}{)}{;}{a}{,}{;}

\newenvironment{refquote}{\bigskip \begin{it}}{\end{it}\smallskip}

\newenvironment{figure}{}


\begin{document}



\newpage

\setcounter{page}{1}



% -----------------------------------------------------------------------------
% -----------------------------------------------------------------------------
{\Large \bf Reply to Associate Editor}
% ---------------------------------------


	\begin{refquote}
	Your manuscript has been assessed by four different referees who agree that you touch upon a very important problem. However, they all feel your manuscript needs substantial improvements. They provide excellent advise on how to revise your manuscript and you should carefully consider all their comments. Overall, you do not acknowledge important past contributions and do not clearly explain the importance of the methodological issues you cover in your study. Importantly, referee 1 feels the real novelty of your study is the consideration of the importance and complexity of choosing prior information in constructing networks. Obviously, this is an issue in the context of Bayesian approaches, which are not necessarily well understood by empiricists, precisely the audience your manuscript should address. Please, follow very closely the suggestions of the referees when revising your manuscript.

	\end{refquote}

	\textbf{R:} We appreciate the thoughtful feedback by the four referees. We take particular note of the comments regarding the structure and grounding of our introduction and have worked to address this. Reviewer 2's comments take a different view of the purpose and scope of the manuscript than the other reviewers; we respect their ideas but feel that some are beyond the scope of the present manuscript. We therefore prioritise comments by the other three Reviewers and hope that, by addressing these, the purpose of our manuscript becomes clearer and our changes will at least partially satisfy Reviewer 2.


% -----------------------------------------------------------------------------
% -----------------------------------------------------------------------------
{\Large \bf Reply to Reviewer \#1}
% ---------------------------------------


	Below I offer my thoughts on this well-written manuscript. Normally I condense my review into a few key points, but I found myself uncharacteristically divided with the paper as presented. At first, the introduction covers ground very well-trod by prior works and doesn’t really address its contribution. I was mostly frustrated by the rehashing of topics made by many previous papers, include many by the authors themselves. But then the paper seems to morph and pick up steam on L248. Rather than reaching towards an amorphous and grandiose “quantitative framework”, the authors briefly zoom into a very concise, and hugely important, question of choosing prior distributions for harnessing information among networks. This has been completely unexplored and merits attention. In ecology more broadly, priors are treated with timidity, as if we couldn’t dare to leverage past work on our current understanding. Personally, I have always felt limited by this standard. I applaud the authors for taking on this challenge, and I would like to see an introduction that merits and debates this question in much more detail. When should we use prior information? How do we go about constructing informed priors for networks? When can we use priors from the data in question? This is sometimes called approximate bayes (http://andrewgelman.com/2016/03/25/28321/), or empirical bayes, perhaps alluded to in the title, but not fully explained in the introduction. Should empirical priors be drawn from hyperpriors or fixed for each interaction? Can we combine qualitative information and quantitative information? This is all exciting ground that will open up new avenues for networks. Walking the reader through flat priors, informed priors and empirical priors is crucial for the importance of analysis to be understood by the reader. The use of empirical priors can be quite controversial, and should be atleast touched on in the discussion.
	I’ve left my comments below on the introduction, but in my opinion, it should be almost wholly rewritten in favor of a more targeted question constructing networks using priors, which is the key novelty of the work and is mostly hidden by the current introduction. I’m not convinced any new analysis needs to be done, which speaks to the conflict between the current introduction and the actual data presented. I see the makings of a really nice paper here, and I’m confident the authors will be able to achieve this text revision.

	Comments
	The introduction flows well and is thoughtfully laid out. Given the enormous amount of literature in the field, it is not possible to provide a full set of citations. However, it seems remiss to not consider Jordano (2016) in the sampling section.

	It’s a bit confusing to lead the reader towards a Bayesian analysis, and then immediately start with the MLE estimate. Why is this needed at all? (L162-186). Just start at 187?
	I think a bit of space is wasted on defining the moments of the beta distribution. This is very googleable, and almost too thorough. I think the space is better utilized aligning the ecological challenge and the quantitative method.
	I’m wrestling with the novelty of the work as presented and whether it reaches the standard for publication at MEE. Certainly, much of this logic is laid out in other works. The authors are well aware of this and have made significant contributions to this topic in the last few years. In particular, the authors should try to be much more explicit about the contribution they hope to offer that hasn’t been covered. I could list atleast two dozen papers here, but some key ones might include:
	All have the theme of detection, occurrence and interaction as a central challenge in constructing
	reliable networks. Several prior papers have offered Bayesian approaches, starting with the individual model by
	While it is true that none of the above papers do the exact same thing as this manuscript, they all
	orbit the same topic. L89 for example, seems to conflict with the two paragraphs immediately following it?
	At the same time, the paper is clear and very readable. I wholly support the method and I think it is something critical that needs to be adopted by the community.

	Having read the paper several times, I think that the authors are trying to cast their net too widely, even starting from the title. From that perspective, they don’t meet the expectation of new insight. However!! I was extremely interested in the brief discussion on using prior knowledge, especially the delicacy of trying to transfer information among networks (L248). My recommendation is to drastically reduce the grandiose attempt at a “quantitative framework for networks” (L1 – L246) and really zoom in on the contribution of transferring information among networks through informed priors.
	Again L278, this is very well covered in concept by recent papers in MEE that include the authors of this work, as well as several by Weinstein, Bartomeus and others.


		Literature suggestions:


	Jordano, P. (2016). Sampling networks of ecological interactions. Functional Ecology, 30(12), 1883–1893. doi:10.1111/1365-2435.12763

	Bartomeus, I. (2013). Understanding Linkage Rules in Plant-Pollinator Networks by Using Hierarchical Models That Incorporate Pollinator Detectability and Plant Traits. PLoS ONE, 8(7), 1–9. doi:10.1371/journal.pone.0069200
	Poisot, T., Cirtwill, A. R., Cazelles, K., Gravel, D., Fortin, M.-J., & Stouffer, D. B. (2016). The structure of probabilistic networks. Methods in Ecology and Evolution, 7(3), 303–312. doi:10.1111/2041-210X.12468
	Poisot, T., Stouffer, D. B., & Gravel, D. (2015). Beyond species: why ecological interaction networks vary through space and time. Oikos, 124(3), 243–251. doi:10.1111/oik.01719
	Graham, C. H., & Weinstein, B. G. (2018). Towards a predictive model of species interaction beta diversity. Ecology Letters. doi:10.1111/ele.13084
	Gravel, D., Poisot, T., Albouy, C., Velez, L., & Mouillot, D. (2013). Inferring food web structure from predator-prey body size relationships. Methods in Ecology and Evolution, 4(11), 1083–1090. doi:10.1111/2041-210X.12103
	Wells, K., & O’Hara, R. B. (2013). Species interactions: Estimating per-individual interaction strength and covariates before simplifying data into per-species ecological networks. Methods in Ecology and Evolution, 4(1), 1–8. doi:10.1111/j.2041-210x.2012.00249.x



% -----------------------------------------------------------------------------
% -----------------------------------------------------------------------------
{\Large \bf Reply to Referee \#2}
% ---------------------------------------

	I do not know what to do with this.

	\begin{refquote}
	The paper by A.Cirtwill and colleagues tackles a very interesting problem, quantifying the uncertainity of interactions, using three levels of nested uncertainity. The authors proposed a bayesian framework to obtain estimate of the parameters of the model and show the results of their approach on a recent dataset. The question of disentangling the true presence/absence of an interaction given a series of observations (replicates) is of primary interest and, seen in this light, the paper is tackling one of the hottest topics in the field.

	The introduction is particularly pleasant to read and the problematic is well stated. However, the paper has also a number of weaknesses that I enumerate in the following :

	1/ lack of references: the fundamental question about predicting interactions given some incertainity has been and is debated in many papers (see the seminal paper by Roger Guimerà and Marta Sales-Pardo, PNAS, 2009). Even if the proposed framework is probably not stated in its exact terms in the literature, bridges between existing approaches (beyond Ecology) and the one in the paper would be very welcomed. An appropriate discussion between points of convergence and difference would help the reader to appreciate the significance and the impact of the proposed methodology.

	2/ network oversight: the approach considers that the interactions are independent, which is not the case in real data. Indeed, the paper is supposed to be “network-oriented“ but the methodology deals with independent interactions (e.g. the variables L,T,X and D introduced lines 62,77,101 are for instance not indexed by ij which could have suggested that the interactions were seen as a whole). Consequently, the mathematical framework remains quite simple and may not reflect nor model the processes governing the actual detection of interactions. Despite of this drawback, the proposed method is interesting but at least the authors should add some discussion on the limits of dealing with independent interactions,

	3/ identifiability: the reader would appreciate that the authors discuss and prove the identifiability of the model. Indeed, the three levels of uncertainity are nested but one could wonder whether different parameters at the three levels can fit the same data with the same quality (i.e. no identifiability).  I would suggest this point to be addressed in a specific section.

	4/ absence of simulations: the authors proposed an original approach based on a bayesian framework. Beyond the identifiability question (see above), the reader can question the power of the method. Usually, a methodological proposal comes with a complete simulation study that is supposed to show  the proposed model is adapted/suitable to the question and efficient to solve it. In the paper in its present form, the authors assume that the reader will be convinced by the results obtained on a single real dataset… In my opinion, the reader can’t be fairly convinced: how powerfull is the method? what can we say about the priors? how much data (n, k) is necessary to expect good results? An so on… Only a well-conducted simulation study can answer these questions.

	5/ results&discussion too short: the reader is surprised to read 27 lines of raw results only… Given the amount of new methodology that is proposed, one can expect a complete description of its pros and cons. Obvisouly, a simulation study (see above) would enrich the results part. Forgetting about the length of these parts, it is still difficult to catch the key messages from the text, and it is not obvious to relate discussion and the obtained results (for instance,from l.343 to l.351).

	6/ lack of rigorous notations: it is possible to understand the approach with the present notations but some improvements can help the reader because there are a series of inconsistencies. Here are some suggestions. The authors chose upper case letters L,T ,X,D for single entities whereas the mathematician is used to expect matrices or constants in upper case. Traditionaly, given species i and j, we would consider interaction L_ij, feasibility X_ij, detectionability D_ij...
	Still, what is T?? not defined, never reused: is it a characteristics on i, on j, on both?
	How can be D<1 on l.169? D=0 or 1 since it says if an interaction is detectable or not...
	Moreover, L is reused later but is the number of interactions (l.100 supp.mat)…
	What is “N” in Figure 4? is it “n”? What is n_ij and k_ij l.266? Never defined, never reused.
	In the beginning of the paper, \lambda is a probability (l.62) but it becomes a random variable (l.164): it would be more consistent to introduce before the notion of hyperparameters  (l.209-213) and to explain we consider that \lambda is drawn into its distribution.

	\end{refquote}

	Minor points:

	-l.62: define formally T
	-l.244: this is weird to mention that the analysis has been done on another dataset but that the results will not be discussed… What is the interest for the reader?
	-l.149: since the authors consider the interactions are independent, there is morally no network in the approach (except the result)
	-l.252: not clear. Why this choice?
	-l.259: not clear
	-l.278: the authors investigate the consequence of uncertainity on network metrics. This question is of peculiar interest, but here it is restricted to the connectance and nestedness. Why these two metrics? Why not others? What is the impact of integrating incertainity into the metrics computation? (we guess the answers but we could expect the authors develop these points).
	l.291: explain better the simulation procedure
	l.306: given the shape of the distribution (exponential), the mean is not an appropriate indicator.
	l.327: what does it mean? How is it possible?

\section*{Reviewer 3}

Reviewer: 3

Comments to the Corresponding Author
Overall, I think this is a very good paper and makes an important contribution.  I think there are three main ways it can be improved:

1.      The simplest – do a better job of acknowledging what else has been done in the field – much of which has actually been done by the authors and their collaborators.  Given that they know there work I won’t point that out (but they are under-siting this literature a bit) but they could consider Graham and Weinstein on-line early 2018 (I realize that this was just out when the author’s submitted but it has some very similar messages) and the empirical example of Weinstein and Graham in Ecology letters 2017.  I don’t think this would in any way diminish the importance of the current paper because I think the challenge in network ecology is to get people to use better quantitative methods.
2.      Provide more clarity in descriptions making sure to consider how terms and ideas have been used previously.  Two things come to mind – the use of terms and distinction between interaction and process uncertainty; and the somewhat contradictory sounding statements about if sampling more is actually good.
3.      The set up (introduction) to the case study is quite odd and somehow seems off topic – or at least not logical based on what the reader has read up to that point.


4. Misunderstanding of our point about sampling effort



	\textbf{R:} We believe that the Reviewer has badly misunderstood the point of our comments on sampling effort. We (like other sane people) would never suggest that researchers intensively sample less intensively. We assumed that field researchers already sample to the maximum extent possible given their time and resource constraints and so a call to "sample more" would have no effect. We now state these assumptions explicitly. Moreover, we believe that increased sampling effort will not reduce all sources of sampling uncertainty. For example, interactions that are not detectable using a particular sampling methodology will not be included by sampling more intensively using that method. Our comments about sampling were intended to draw attention to this and other issues, not to say that sampling is bad. Indeed, we point out that a high degree of sampling is necessary to detect rare species and to gain confidence that any two co-occurring species do not interact. Sampling is a valuable tool that happens not to solve the problems we are most interested here. To be absolutely clear about this to future readers we have added a box (Box 1) where we discuss why sampling alone will not solve problems about uncertainty. We hope that this box will make our (pro-sampling) stance clearer to the Reviewer and to future readers.



	Line 94.  I find this a very odd argument against sampling multiple times.  So, if you sample more you might see more then understand the system less?  The logic does not make sense to me. 

	\textbf{R:} Of course sampling more does not lead to less understanding. It can, however, reveal knowledge gaps (as when a rare species is detected, revealing our ignorance of its interactions). This is not an argument against multiple sampling but a statement that increased sampling is likely to lead to the addition of more "false zeros" as species about which we know little are added to the network.

	Line 107.  There is a lot of literature in wildlife ecology emphasizing the importance of repeat sampling for estimating detection probability.  Is this just wrong?  If so this does need more explanation in the context of this literature.

	\textbf{R:} I do not know how the Reviewer assumes that we are against repeat sampling when this is, in fact, one of the core parts of our approach. Our point here is that some species (e.g., cryptic or difficult-to-identify species) and interactions (e.g., brief visits of pollinators to plants) are difficult to detect and may be missed even with multiple sampling. If a rare species is only detected in one sampling round, then the repeated sampling necessarily tells us little about its interactions. For species which are frequently detected, we absolutely learn about the detection probability of their interactions with repeated sampling.

	Line 130.  The ideas here are largely a repeat from those above.  Further, while it may be true that we can sample too much – I think it is dangerous to suggest we should sample less – do the authors think that sampling across most network studies is sufficient (the opposite is stated in the discussion)?  I wonder if the argument is a bit more statistical than biological.

	\textbf{R:} We agree that it is dangerous to suggest that anyone sample less, that is why we have not done so. We argue only that the amount of sampling required to be confident that co-occurring species do not interact is large -- likely beyond what is feasible for speciose systems given the limited resources available to many researchers. Sampling across most network studies is almost certainly not sufficient, but data which would allow readers to judge this (i.e., number of times each species is observed, independent of numbers of observed interactions) is not generally provided. We chose to be more charitable than the Reviewer and assume that those compiling empirical networks are sampling as much as possible given the constraints of their system and available resources. Infite sampling would, of course, be the best option but it is unlikely that researchers will be able to perpetually increase sampling.

	Line 144.  I am much more comfortable with this statement than how sampling effort should be considered and I would suggest moving this up and then explaining how even if we sample well we need to consider uncertainty for the reasons you mention in the previous section.  Please note that you make the same point in your discussion on line 338.


	Line 319.  It is clear that these are large samples but if interaction among co-occurring species was greater than you need lower samples – correct?  Should this be made clear?  Otherwise, we maybe can just never get enough data!

	Line 338 to 342.  Throughout I am confused if the authors suggest more sampling is good or bad…  At the end of the paragraph it seems it isn’t so great to sample more because you end up with these annoying “0” values…??????



Below I provide specific comments


	Line 7.  It is certainly true that most networks are snapshots in time but there is a growing realization – including some nice examples (including those by the authors) that address this issue.  Maybe what you want to say is that we know that networks should not be considered static but we lack the tools – or ability to collect data – or whatever you think is the cause – to correct this situation. 

	\textbf{R:} We have softened this statement to say that empirical networks are "often limited" and hope that this rephrasing will not offend those who are working to include variation in their descriptions.

		\begin{quotation}
			Despite this additional information, empirical descriptions of ecological networks are still often limited by a lack of data or tools to adequately incorporate variation in interactions into networks.
		\end{quotation}

	Line 23.  Delete the “because” (you have one in the sentence already).

	\textbf{R:} Done.

Line 30.  There are several conceptual and empirical attempts to consider detection probability – why aren’t these cited (work by Bartomeus comes to mind).

Line 36.  I certainly see value in the approach and the general ideas of these authors in particular but they should acknowledge the work that has been done by themselves and others!

Line 86.  So, similar factors (i.e., trait matching) can be used to detect different kinds of uncertainty.  Should this be stated?  You might also consider a different word that process as “observation models” and “process models” are commonly used terms in Bayesian stats and in this case “process” often considers what you call “interaction”.  There are some interesting to attempts to model co-evolution that consider similar ideas (i.e., a barrier to interaction and then trait-matching; but maybe outside of the scope of this MS). While I appreciate that your naming has merit, I think if your hope is that more ecologists use these methods then the terminology should be consistent across papers/approaches to the greatest extent possible.

	Line 73, 87.  Maybe the idea that some uncertainty is “inevitable” should not be repeated in multiple sections.  Either find a different way of saying it – or state it in your introductory paragraph and then don’t keep repeating.

	\textbf{R:} We have rephrased the paragraph about detection uncertainty. The last line now reads:
		
		\begin{quotation}

			Until these gaps are filled, some interaction uncertainty will remain.

		\end{quotation}

Line 89.  In Weinstein and Graham (Ecology Letters) the processes modeled were trait matching (which you place in interaction uncertainty) and abundance (which is perhaps your process uncertainty??).  I find myself quite confused based on how you describe things in this MS (See text above) and how you cite the literature.

Line 142.  But shouldn’t you model the probability of two species co-occurring?  The authors have a paper doing this…. Is that not a good idea?

Line 148.  Please note that a paper with a very similar aim was recently published (Graham and Weinstein).  This does not to diminish the value of this study in any way, as multiple perspectives on quantitative solutions in network ecology are urgently needed….. but the other work should be considered. More generally, the introduction does not really do justice to what has been done already – even the authors own work.  The problem in my view is that the community using better statistics – of which the authors are very important members – has not managed to communicate to most people studying networks and better methods are simply not being adopted.  The current paper is a very nice attempt to show the broader community that the Bayesian approach is not so complex and has many advantages.  I think this is a very important message and is what makes this paper an important contribution.  Further, the approach to parse out different types of uncertainty and suggestions for how to come-up with informative priors – is very useful.  The theory/method is well developed overall and provides an example – which is great.

Line 121.  I think one of the reason non-Bayesians’ are not comfortable with the approach is that if you put in prior information on something like trait matching and test for trait matching and then discover trait matching is important – what does it mean?  Maybe this is too obvious (I realize there are tons of papers on how to choose priors) but I wonder if it is worth explaining (would a box or something be worthwhile?  I leave it up to the authors/editor)?  Maybe the issue could be acknowledged and an appropriate paper cited?

Line 235.  Correct redundancy.

Line 240.  I found the justification for the system a bit odd.  Is the goal to show the gaps in sampling or to apply the model described thus far to consider the 3 different types of uncertainty outlined?  At this point the reader is confused about the goal…

Line 249.  The statement about training data is redundant with what is stated above – adjust writing.

Line 288.  I am pretty sure this idea is in the literature several times and should be cited.  Isn’t this the same as drawing a probability of interaction from a distribution where the distribution of each interaction is an output of the Bayesian method?

Line 295 to 302.  I would think that this second step (i.e., filtered networks) is to explore sampling not to determine how networks will be influenced by uncertainty?  Do you need the filtering step to evaluate uncertainty (as written it seems that this is the case)?


Line 352 – 358.  Citations are needed in this section as this has been suggested before.

Line 361.  I find the statement that 30-50 individuals need to be evaluated a bit strong.  This statement is based on this study – are all systems like the gall system?  If species are specialized or bound to interact for some other reason (i.e., co-occurring when there are few other resources) wouldn’t you need to observe fewer individuals?  The point of the method is that you can estimate how many individuals are needed…. This is great!  But given that this can be estimated from any given system why give a value from one system and state that is what is required?  It defeats the point of the very nice method proposed…??

I hope this review is helpful.  


\section*{Reviewer 4}

Comments to the Corresponding Author
I have been surprised how little that sampling effects have been formally considered in the analysis of networks, so I welcome this contribution that provides a theoretical framework. Overall, I felt that this contribution was really helpful, and that it provides both a theoretical framework and a practical example.

However, I felt that the structuring of the paper made it a challenge to understand and apply. It felt like the paper itself had been written in different sections (by different authors?) with not enough links between the sections. I also felt that there was a strong theoretical component, which is important, but there needed to be a stronger link with the issues around empirical ecology, as explained below.

Introduction – T. Poisot and colleagues have done a lot of relevant work in this area, but I felt there was undue reference to their work in comparison with other relevant work. There were some obvious papers (e.g. sampling by Jordano, forbidden links by various others) that were not cited. A broader perspective on the literature would, I think, help the authors appreciate the value of their paper and the need to ensure that it is fully understandable and of practical use.

L36ff The authors should note the value of weighted metrics in taking account of sampling biases.

L45 It is a shame not to give some idea of how the method could be extended (or is this the intention of the authors?). For empirical analysis of networks it is rare to use binary networks given the value of weighted networks for taking account of sampling biases. Indeed, I am much less concerned about whether an interaction never occurs, and much more concerned with the frequency of occurrence.

\textbf{R:} Unfortunately, the use of binary networks is not as rare as the Reviewer thinks, particularly in food webs where it is often difficult to quantify interaction strengths or frequencies in the field. We absolutely agree that weighted networks are much more valuable when they are available but did not wish to exclude those without such data. [[Do we want to add a bit of non-binary stuff or just make it clearer that we are talking about feasibility and that probabilities of occurrance could be layered on later?]]


L50 This is a nice and clearly explained description of the problem that many network ecologists ignore, or are unaware of.

Section from L50. This section is valuable, but had very weak links with the empirical analysis later in the paper. For instance in the example of the gall-formers, there was not formal discussion about the different sources of uncertainty. [[So we need to state more clearly that we cannot distinguish these after the fact.]]

L76ff ‘Process uncertainty’ is not defined. The simplistic example of interaction uncertainty (fish eating a cactus) does not help the reader understand the detail of what is meant by ‘interaction uncertainty’. The two examples of ‘local constraints’ (weather and habitat) are operating on completely different temporal scales (note – it is defined differently in L109). I would regard the issue of weather as much closer to detection uncertainty (e.g. for a pollinator, it would be likely to be interacting if the weather was better), whereas the issue of habitat is closer to the authors’ ‘interaction uncertainty’ (the species don’t interact because they are not, or never, co-occurring in a habitat). Also if the species occur in different habitats then with an appropriate spatial scale of sampling then they do not co-occur – so the issue seems redundant. [[I suspect that the authors are thinking of a meta- or master network in much of their study, but this is not clear (or I have misunderstood and it needs to be explained better), i.e. the predicted presence of an interaction is not the presence of an interaction at the local site (taking detection and process uncertainty into account) but the ability to assess ‘interaction uncertainty’.]][[This is a bit separate from their issue]]


\textbf{R:} It seems our descriptions of interaction uncertainty and process uncertainty were both unclear, as the Reviewer has badly mixed them up. We have expanded the paragraph introducing the different levels of uncertainty to include plain-language questions addressed by each level of uncertainty; hopefully this will help to avoid confusion at the outset.
	
	\begin{quotation}

		As a conceptual guide to the factors affecting this process, we describe three nested levels of uncertainty: interaction uncertainty, process uncertainty, and detection uncertainty (Fig.~\ref{conceptual_fig}). These levels roughly address the questions: "Could species $i$ and $j$ interact?", "Do they interact at this site/time?" and "Do we observe the interaction?".

	\end{quotation}

We have also added a paragraph to explain the interaction uncertainty explicitly \emph{does not} depend on co-occurrence; only on species traits. The Reviewer is not correct that habitat belongs in interaction uncertainty as species which do not currently share a habitat might still interact if one is introduced into the other's habitat (for example). We now state this explicitly.

	\begin{quotation}

		Note that interaction uncertainty does not depend upon whether species $i$ and $j$ actually co-occur. Ecologists are often interested in predicting which species are likely to interact following an introduction~\citep{} or range shift~\citep{}. In such cases, models based on trait matching can give some information about $\lambda$ while past co-occurance will, of course, not yield much insight.

	\end{quotation}

Process uncertainty covers factors which prevent an interaction which \emph{can occur} from actually occurring at particular places/times. This includes factors which mean that two potentially interactive species do not co-occur (e.g., habitat requirements) as well as factors which reduce interaction probability between co-occurring species (weather, abundances, etc.). We have rephrased the first line of this section to make this explicit.

	\begin{quotation}
		
		Assuming that an interaction is feasible (i.e., $L=1$; species traits do not prohibit an interaction if they should co-occur), an interaction still not occur at a particular place or time. 

	\end{quotation}

Weather likely affects both process uncertainty and detection uncertainty. For pollination, pollinators are much less likely to interact in the rain (process uncertainty) and the lower visibility may mean that some interactions which do occur are missed (detection uncertainty). We have added a paragraph in the section describing detection uncertainty to make it clear that some external factors can affect multiple types of uncertainty. 

	\begin{quotation}

		Note that some exteral factors can affect both process and detection uncertainty. For example, rare species are less likely to interact at a given site due to neutral processes (process uncertainty)~\citep{}. Interactions involving rare species are also less likely to be detected (detection uncertainty), unless more sampling effort is focused on rare than common species~\citep{}. Similarly, bad weather can make interactions such as pollination less likely by reducing pollinator activity~\citep{}. For those pollinators which do venture forth in poor conditions, bad weather might also reduce visibility and therefore increase detection uncertainty.	

	\end{quotation}

The Reviewer is correct that increasing sampling effort to include more interaction-promoting weather, sites, etc. will reduce process uncertainty. It would also be possible to gerrymander the spatial scale of sampling such that non-interacting species do not co-occur, but we would not advocate selecting a study site to avoid particular species. The appropriate spatial and temporal scale of sampling is an important question, but it is not the question we are addressing here. Instead, we have added a line to include the Reviewer's point that expanded sampling can help to address process uncertainty and left it at that.

	\begin{quotation}

	Expanding sampling to cover a broader spatial or temporal range (e.g., sampling in a variety of microhabitats or during a variety of weather conditions) will also help to reduce process uncertainty if resource constraints permit.

	\end{quotation}



L76ff Different interactions occur at different temporal scales, so it is important to consider where the role of this study lies. For instance, a pollinator-flower interaction is quick (and even the evidence of it, e.g. pollen on the insect, is not long-lasting) whereas an active gall-former interaction could be present for a couple of months and could be detected for even longer (on senescing leaves).


\textbf{R:} Long-term interactions may be easier to detect, but we do not see how this affects process uncertainty. Spatial variation in interaction probabilities will affect all interaction types, as does variation in abundances between years, so we cannot conceive of any interactions that are exempt from this issue. The timescale of an interaction (and the evidence it leaves) does affect detection uncertainty, so we have added a note in that section. Long-term interactions may fail to be detected, however, and we now add an example to make it clear that detection uncertainty applies to even long-term interacitons such as gall-forming.

	\begin{quotation}

		Some types of interactions will have higher detection uncertainty than others (e.g., pollination or predation often take place very quickly while parasitism can last for months or years). Even long-term interactions, however, can be missed, especially if species are difficult to identify or if not all individuals of a species share the interaction. Parasites, for example, are often concentrated in only a few individuals~\citep{Lagrue2017}. If these infected individuals do not happen to be included in a sample, their interactoins will be missed.

	\end{quotation}


I was struggling to really understand what the authors were really trying to model (partly because of the lack of a link between the theoretical and empirical parts of the paper). Are the authors trying to estimate L (the probability that an interaction is feasible or not – I would describe this as a ‘master’ or meta- network), or is it that they are trying to estimate X|L (the probability that the interaction locally occurs)? I would be interested in both, but the second aspect seemed to be ignored in the gall-former example.

L136 This is a completely different issue, that you do not address in the paper.

L147 Your Bayesian approach seems important, especially the prior information, and I would have appreciated it being explained more simply.

L155ff I think it would be helpful for this section to be integrated with the previous section. It relies on the reader putting quite a lot of work in to understand the link between the two.

L209ff This seems a really important set of ideas, but they are exemplified very briefly and very simplistically in the paper (L262, L310). I would have appreciated a much clearer explanation and example.

L241 I wonder if you could phrase this more positively as ‘solutions’ rather than just ‘difficulties’?

L283 This seems a very important point and should be explained/expanded more clearly.

L291 I have wondered about this – does assuming a probability of 1 for observed interactions create a bias? The observed interactions are a stochastic set of the possible interactions, so by treating observed interactions as 1 and other (equally likely?) unobserved interactions as <1 will surely bias the network metrics. I’m not sure of the solution and would welcome thoughts on this in the paper.

L298 These seem arbitrary. It is not clear what is filtered, or its justification – is it the unique interactions, unique interaction per site or samples? How do these compare to typical sampling effort?

L321 This is a really important and interesting set of results that are worthy of greater discussion. The issue of metrics from sampled networks is important, but gets lost in the paper. Please expand upon this.

Overall, I fear that this review is rather meandering, but I think that reflects some of the challenges I have with the paper. Currently I do not think the paper would be used by empirical network ecologists because they either would not understand its relevance or would not be able to apply it practically. This would be a great shame. I hope that the comments are clear enough to allow the authors to improve the paper – the paper has the potential to be important and very useful.
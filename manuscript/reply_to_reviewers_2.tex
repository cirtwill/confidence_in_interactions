\documentclass[12pt]{letter}

\usepackage[britdate]{LiU-letter}
\usepackage{times}
\usepackage{letterbib}
\usepackage{geometry}
\usepackage[round]{natbib}
\usepackage{graphicx}
\geometry{a4paper}
\usepackage[T1]{fontenc}
\usepackage[utf8]{inputenc}
\usepackage{authblk}
\usepackage[running]{lineno}
\usepackage{amsmath,amsfonts,amssymb}
\usepackage[margin=10pt,font=small,labelfont=bf]{caption}

%\usepackage{natbib}
% \bibpunct[; ]{(}{)}{;}{a}{,}{;}

\newenvironment{refquote}{\bigskip \begin{it}}{\end{it}\smallskip}

\newenvironment{figure}{}


\begin{document}



\newpage

\setcounter{page}{1}


% -----------------------------------------------------------------------------
% -----------------------------------------------------------------------------
{\Large \bf Reply to Associate Editor}
% ---------------------------------------


	\begin{refquote}
		
		Thank you for the extensive and detailed reply to the comments of the referees. Your revised manuscript was sent back to one of the referees who reviewed the original version. As you will see, he is very positive about the revised version and has now signed his review. The only outstanding issue is the addition of some discussion about how to detect potential problems with the priors. I agree with this suggestion as one of the most common criticisms of Bayesian methods is precisely the use of priors.

	\end{refquote}


	\textbf{R:} We appreciate the additional comments from the Referee, and agree that the definition of adequate priors can be quite a challenge. We are therefore quite happy to add to our discussion about how to detect potential problems. Note, however, that after making the additions requested by the previous round of reviews our manuscript was very close to the standard word limit for \textbf{Methods in Ecology and Evolution}. To meaningfully expand our discussion of how to detect problems with priors we must therefore ask for more space-- a solution also explicitly recommended by the Reviewer. We have kept our addition to the discussion as succinct as possible, and the revised manuscript is now approximately 7433 words. We hope that this additional length is justified by our efforts to address the Reviewer's remaining concerns.


	Thanks again to the Associate Editor and Reviewer for their comments. We believe that the manuscript has been much improved by their feedback.


\newpage


% ----------------------------------------------------------------------
% ----------------------------------------------------------------------
{\Large \bf Reply to Reviewer \#1 (Ben Weinstein)}
% ---------------------------------------

	\begin{refquote}

		I am reviewer 1 from the first submission. 

		The text is remarkably improved. It is focused and clear. The authors do a good job of identifying the contribution of their work. I appreciate the authors willingness to rewrite much of the paper based on the suggestions of the reviewers. 


		Sincerely, 

		Ben Weinstein

	\end{refquote}


	We thank the Reviewer for his previous and further feedback and address each point (prefaced by \textbf{R:}) below.

	1. Re-emphasise Figure 3. 

		\begin{refquote}

			Figure 3 is fun and deserves a bit more central place in the discussion. 

		\end{refquote}


		\textbf{R:} We thank the Reviewer for his complement to Figure 3. Although we wish to keep any additions as brief as possible, we are happy to add a short reference to Figure 3 in the discussion as part of an example of how to use our simple Bayesian framework when thinking about insects moving into new communities (e.g., as climate changes). We hope that this puts the figure in a more central place, as intended. We also now include a similar figure demonstrating the effect of a change in prior on the posterior pdf's obtained for the \emph{Salix}-galler networks and hope that this also helps to emphasise these results.


		Lines 379-389:

		\begin{quotation}

			Such approaches are particularly useful when considering interactions involving species entering new ranges due to climate change or introductions. Our framework could, for example, be used to predict the probability of interaction between a galler and parasitoid at a new site on the frontier of their ranges. If the species have not been observed co-occurring at other sites, we would expect them to interact with a probability of approximately 0.1 rather than assuming that they will not interact because they have not been observed interacting elsewhere (Fig. 3). Understanding species' interactions in novel or changing communities is important for a variety of conservation questions~\citep{Bartomeus2013,Gravel2013}, and a Bayesian approach using a trait-matching model or data from species' current ranges could help us to anticipate how species will integrate into new communities. 

		\end{quotation}


	2. Emphasise the drawbacks of empirical Bayes


		\begin{refquote}

			I think the authors miss one key discomfort with empirical Bayes. It seems vaguely anti-ecology, because it assumes that we can L289 : “use information from abundant and easily-sampled species to predict interaction probabilities for species which were observed only rarely.”  A common goal in network ecology is to describe interaction frequencies by using niche-based proxies (phylogeny, traits, etc). If it was a simply a matter that most networks had the same degree distribution, then predicting networks would be much easier than it is. In fact, we are really bad at predicting pairwise interactions (see Olito and Fox), even when we get the emergent structure of the network correct. Yet empirical Bayes assumes a kind of commonality among all networks to help narrow the posteriors. I think the authors need to speak to this concern.

		\end{refquote}


		\textbf{R:} We appreciate but do not fully share the Reviewer's negative view of empirical Bayes. Perhaps this is a matter of perspective: in a system where a great deal of information about the traits structuring interactions is known and trait values are available for most species, trait-based models are indeed often a good way to predict interactions (although far from fool-proof, as the Reviewer points out). This is why we began our section on informative priors with trait-based models and only introduce empirical and approximate Bayes as solutions where trait information/models are lacking or where over-fitting is a strong concern. Most of the authors of this manuscript are more used to such relatively information-poor systems. In these cases, we agree that empirical Bayes is not perfect, but argue that it is much better than nothing (or an uninformative prior, as long as a somewhat similar network is available).


		% [[Currently one vote for removing this paragraph and one vote for keeping it and strengthening the point]]
		An uninformative prior like the Jeffreys prior gives a maximum likelihood estimate of 0.5 for the probability of an interaction between two species which have never been observed together. Based on our understanding of ecological systems, this seems highly unlikely. Certainly species differ in many ways, but they also do have some similarities, such as limited foraging time and somewhat consistent morphologies. Because of behavioural, morphological, or temporal limits to the number of partners with which a given species can interact, most ecological networks are sparse and most well-sampled species interact with only a small subset of their potential partners (extreme generalists in highly nested mutualistic networks being a notable exception). We can therefore reasonably expect that a species entering a new habitat, for example, will interact with far fewer than half of the available partners. If we do not know which traits strongly affect interactions involving the focal species, or if those traits are not known for all of the species in the network, we cannot use a trait-based model to predict interaction frequencies. This leaves empirical and approximate Bayes as the only viable options, unless we are willing to revert to the assumption that unobserved interactions never occur. Certainly empirical Bayes is not perfect, but we are confident that it is still gives more reasonable expectations than ignoring the ecological networks literature completely.


		Moreover, while we agree that empirical Bayes does not take differences in niche proxies into account, this may not be crucial in all studies. Researchers are often more interested in the overall structure of the network or in some emergent property such as stability than in the relationships between niche proxies and interaction frequencies. In such cases, the number of additional interactions that we expect to be observed with further sampling can be more important than the exact identity of those interactions, and an empirical Bayes approach is reasonable if a good trait-based model and trait information or not available. 


		Finally, we disagree with the Reviewer's contention that ``empirical Bayes assumes a kind of commonality among all networks". This is not so; empirical Bayes only assumes some commonality between the focal network and \emph{those used to construct the prior}. Empirical Bayes does not imply anything about networks not referred to in the prior. We have revised the text to emphasise the need to use similar networks to construct an empirical Bayes prior and note that \emph{Appendix S7} demonstrates the pitfalls of basing an empirical Bayes prior on a network with little commonality with the focal system. 


		We believe that including an expanded reference to \emph{Appendix S7} as suggested below will help to address some of these concerns by demonstrating that empirical Bayes should \emph{not} be based on networks which have major differences from the focal system as this will lead to unreasonable predictions (in our case, although the taxa and interaction types involved were similar, the taxonomic resolutions in \citet{Barbour2016} and \citet{Kopelke2017} were very different). To reinforce this point and make the role of empirical Bayes as a solution in the absence of trait-based models and/or trait information, we have made several minor revisions to the section introducing informative priors. We hope that these changes will be sufficient to illustrate the drawbacks to empirical Bayes without being overly discouraging for those who lack the necessary information to construct more detailed informative priors.


		In lines 258-262 we state that empirical Bayes is an option if information about traits is lacking but that an empirical Bayes prior does not vary between species pairs:


		\begin{quotation}

			Alternatively, if information on the traits affecting interaction probabilities is lacking, one can develop a prior based on the properties of a set of published networks (``empirical" or ``reference Bayes"~\citet{Spiegelhalter2000}). This type of prior does not vary between species pairs but entails only one assumption: that the focal network has similar structural properties to those of some published networks. 

		\end{quotation} 


		In lines 264-266 we explicitly state that the networks used to build an empirical Bayes prior should be similar to the focal system:


		\begin{quotation}

			The distribution of published connectances can be used to predict the distribution of interaction probabilities in a new network that is similar to those used to construct the prior (same type of interaction, similar taxonomic resolution, habitat, etc.).

		\end{quotation}


		In lines 269-272 we reiterate the fact that an empirical Bayes prior does not vary between pairs of species and that the networks used to build the prior should be similar to the focal network. We also acknowledge the possibility that published data are derived from under-sampled data and therefore will underestimate interaction probabilities


		\begin{quotation}

		Note that here the prior distribution will be the same for all interaction probabilities. It is also possible to interpret other network properties (e.g., species degrees) in a similar network(s) as interaction probabilities and use these to create a prior (worked example in \emph{Appendix S4}).

		\end{quotation}


		Finally, we discuss caveats to the empirical Bayes approach in lines 272-281, including the possibility that priors based on substantially different systems will be misleading (with reference to \emph{Appendix S7}) and that published data may reflect under-sampled data:


		\begin{quotation}

			When using network summary statistics or degree distributions, it is important to consider whether the available information reflects what we believe about the network. Given the likelihood that most published networks are undersampled~\citep{Jordano2016}, it may be wise to include only the highest-quality networks available in a prior. Likewise, only the degree distribution for a similar system is likely to be a reasonable prior for a focal system. For example, if the focal system is large and contains a diverse array of species, it would be inappropriate to use a prior distribution drawn from a small network describing closely-related taxa (demonstrated in \emph{Appendix S7}). Similarly, data from a network where nodes are resolved to different taxonomic levels than the focal network are likely to have quite different properties and would not be an appropriate prior.

		\end{quotation}




	3. Include material from \emph{Appendix 7} in the main text

		\begin{refquote}
		
			My final lingering concern is that something like appendix 7 seems important and should be included in the text? Was this done because of space limitations? I think the authors can appeal to the editor that the paper merits the extra space. As a compromise there could to be at least a paragraph on what can go wrong and what that looks like. I think a figure would be useful here. If the informed/empirical priors are too narrow, the data will not really move the posterior. The authors certainly acknowledge it in various places, but it needs to be clearer to the reader.

		\end{refquote}


		\textbf{R:} This material was indeed placed in an appendix because of space constraints. We have added a figure showing the posterior distributions obtained from both priors and a paragraph summarising \emph{Appendix 7}. As we do not wish to be too demanding with regard to extra space (and our manuscript was essentially at the standard length limit after the last revision), we hope that these smaller additions will be enough to emphasise the potential for spurious results after poor prior choices. Also note that the results for the galler-parasitoid community were somewhat similar. In order to more clearly demonstrate the potentially large effects of prior choice, we therefore showcase the results from \emph{Salix}-galler networks in this paragraph.


		Lines 390-405:

		\begin{quotation}

		  Note that, as in all Bayesian analyses, our results do depend on the prior chosen. To demonstrate this, we repeated our analyses using a prior derived from a study of gallers found on several genotypes of \emph{Salix hookeriana} and the parasites which emerged from them (\citealp{Barbour2016,Barbour2016Dryad}; \emph{Appendix S7}). Although the study system is similar to that in~\citet{Kopelke2017}, the network is quite different due to using different genotypes of a single \emph{Salix} species rather than several \emph{Salix} species as the basis for sampling. While this had a relatively small effect on our expectations for the galler-parasitoid community (\emph{Appendix S7}), the prior based on \citep{Barbour2016} resulted in very high probabilities of interaction between \emph{Salix} and galler pairs that were not observed interacting (Fig. 4). No amount of additional sampling would allow us to conclude that a given \emph{Salix}-galler pair did not interact with a threshold interaction probability of 0.01. This is reasonable for the situation described in~\citet{Barbour2016}, as it is very likely that gallers which can interact with one \emph{S. hookeriana} genotype can interact with most others, but is not reasonable for the more diverse community in~\citet{Kopelke2017}. This demonstrates the importance of conducting a ``sanity check" on the posterior distribution obtained from any given prior.

		\end{quotation}



	3. Reduce citations by Graham and Weinstein 

		\begin{refquote}

			To be perfectly honest, I feel more than a bit uncomfortable with the amount that Graham and Weinstein are cited in the introduction. I count 18 times. It would be more than sufficient to cite the work in the appropriate places and move on. I understand that the reviewers asked for a broader sense of the literature, but I think the authors have gone a bit overboard. I don’t know how often a reviewer asks the authors to cite himself less, but I assure you the reason that the paper is improved is not because I am cited nearly twenty times.


		\end{refquote}

		\textbf{R:} We have removed several of the citations to Graham and Weinstein/Weinstein and Graham, and thank the Reviewer for the opportunity to reduce our word count slightly. We had added particularly many citations of these papers because a couple of the Reviewers mentioned Graham and Weinstein as references that we had missed, and so we hope that the reduced number of references in the present revision still does justice to your work.



\clearpage

    \bibliographystyle{ecol_let} 
    \bibliography{manual} % Abbreviate journal titles.



\end{document}
\documentclass[12pt]{letter}

\usepackage[britdate]{LiU-letter}
\usepackage{times}
\usepackage{letterbib}
\usepackage{geometry}
\usepackage[round]{natbib}
\usepackage{graphicx}
\geometry{a4paper}
\usepackage[T1]{fontenc}
\usepackage[utf8]{inputenc}
\usepackage{authblk}
\usepackage[running]{lineno}
\usepackage{amsmath,amsfonts,amssymb}
\usepackage[margin=10pt,font=small,labelfont=bf]{caption}

%\usepackage{natbib}
% \bibpunct[; ]{(}{)}{;}{a}{,}{;}

\newenvironment{refquote}{\bigskip \begin{it}}{\end{it}\smallskip}

\newenvironment{figure}{}


\begin{document}



\newpage

\setcounter{page}{1}


% Changes: reducing discussion of different sources of uncertainty, increasing description of priors, added a bunch of refs, trying to frame our study more clearly. Changed filtering to reflect all uncertainty, not just detection (more accurate, hopefully less confusing). No box needed, just ``Why sample'' section.


% Ask for help: how would we expand this to interaction frequencies? Still with the Bernoulli? Worthwhile to add a demo of Jensen's inequality?
% To do: expand discussion of our results as much as possible given space constraints.
% -----------------------------------------------------------------------------
% -----------------------------------------------------------------------------
{\Large \bf Reply to Associate Editor}
% ---------------------------------------


	\begin{refquote}
		
		Thank you for the extensive and detailed reply to the comments of the referees. Your revised manuscript was sent back to one of the referees who reviewed the original version. As you will see, he is very positive about the revised version and has now signed his review. The only outstanding issue is the addition of some discussion about how to detect potential problems with the priors. I agree with this suggestion as one of the most common criticisms of Bayesian methods is precisely the use of priors.

	\end{refquote}


	\textbf{R:} We appreciate the additional comments from the Referee, and agree that prior detection can be quite a challenge. We are therefore quite happy to add to our discussion about how to detect potential problems. Note, however, that after making the additions requested by the previous round of reviews our manuscript was very close to the standard word limit for \textbf{Methods in Ecology and Evolution}. To meaningfully expand our discussion of how to detect problems with priors we must therefore ask for more space. We have kept our addition to the discussion as succinct as possible, and the revised manuscript is now approximately XXXX words. We hope that this will be an acceptable length for publication in \textbf{Methods in Ecology and Evolution}.


	Thanks again to the Associate Editor and Reviewer for their comments. We believe that the manuscript has been much improved by their feedback.


\newpage


% ----------------------------------------------------------------------
% ----------------------------------------------------------------------
{\Large \bf Reply to Reviewer \#1 (Ben Weinstein)}
% ---------------------------------------

	\begin{refquote}

		I am reviewer 1 from the first submission. 

		The text is remarkably improved. It is focused and clear. The authors do a good job of identifying the contribution of their work. I appreciate the authors willingness to rewrite much of the paper based on the suggestions of the reviewers. 


		Sincerely, 

		Ben Weinstein

	\end{refquote}


	We thank the Reviewer for his further feedback and address each point (prefaced by \textbf{R:}) below.

	1. Re-emphasize Figure 3. [[done]]

		\begin{refquote}

			Figure 3 is fun and deserves a bit more central place in the discussion. 

		\end{refquote}


		\textbf{R:} We thank the Reviewer for his complement to Figure 3. Although we wish to keep any additions as brief as possible, we are happy to add a short reference to Figure 3 in the discussion as part of an example of how to use our simple Bayesian framework when thinking about insects moving into new communities (e.g., as climate changes). We hope that this puts the figure in a more central place, as intended. We also now include a similar figure demonstrating the effect of a change in prior on the posterior pdf's obtained for the \emph{Salix}-galler networks and hope that this also helps to emphasize these results.


		Lines XX-XX:

		\begin{quotation}

			Such approaches are particularly useful when considering interactions involving species entering new ranges due to climate change or introductions. Our framework could, for example, be used to predict the probability of interaction between a galler and parasitoid at a new site on the frontier of their ranges. If the species have not been observed co-occurring at other sites, we would expect them to interact with a probability of approximately 0.1 rather than assuming that they will not interact because they have not been observed interacting elsewhere (Fig.~\ref{Salix_pdfs_cdfs}). Understanding species' interactions in novel or changing communities is important for a variety of conservation questions~\citep{Bartomeus2013,Gravel2013}, and a Bayesian approach using a trait-matching model or data from species' current ranges could help us to anticipate how species will integrate into new communities. 

		\end{quotation}


	2. 


		\begin{refquote}

			I think the authors miss one key discomfort with empirical bayes. It seems vaguely anti-ecology, because it assumes that we can L289 : “use information from abundant and easily-sampled species to predict interaction probabilities for species which were observed only rarely.”  A common goal in network ecology is to describe interaction frequencies by using niche-based proxies (phylogeny, traits, etc). If it was a simply a matter that most networks had the same degree distribution, then predicting networks would be much easier than it is. In fact, we are really bad at predicting pairwise interactions (see Olito and Fox), even when we get the emergent structure of the network correct. Yet empirical bayes assumes a kind of commonality among all networks to help narrow the posteriors. I think the authors need to speak to this concern.

		\end{refquote}


		\textbf{R:} We agree that pairwise interactions are often difficult to predict and that different networks can have quite different degree distributions (or whichever other network property you like).  We disagree, however, that empirical Bayes assumes commonality among all networks. Rather, we argue that given some \emph{a priori} similarity among networks (e.g., similar taxa, similar habitats, similar interaction types, similar numbers of species, etc.), we can use that similarity to set bounds on our expectations about pairwise interactions. Including clearly dissimilar networks in the set of networks used to create an empirical Bayesian prior would not be reasonable. We do point out some of the potential for false results due to the use of very different networks for the prior and posterior in our paragraph on degree distribution, and have added a few more references to the pitfalls of empirical Bayes.


		We would also like to add that we began our discussion of informed priors with trait-based models because these models are likely to give the best results in terms of which specific interactions are most likely. We introduce empirical Bayes for those situations where researchers do not have a strong idea of the "niche-based proxies" structuring interactions in a given site. In such a situation, we still maintain that using information available from well-sampled species or other networks is likely to be better than nothing. For example, we know that empirical networks in general tend to be rather sparse and the well-sampled species in a community probably do not interact will all possible partners, so it is not reasonable to assume that every unobserved interaction truly occurs (leaving a completely connected network) except in extremely unusual circumstances. Granted, this is quite an extreme example, but it illustrates the point. 


		The Reviewer is quite correct that an empirical Bayes framework does not tell us which interactions are most likely to truly occur (assuming equal observations of co-occurrences). While there are many studies which aim to identify the exact set of interactions which occur, there are also others that are more concerned with the structure of the network as a whole. In such cases, and absent the information necessary to build a more specific model, we feel that empirical Bayes can still be a useful tool. We hope that our additions to this section will now instill an appropriate level of caution in readers without making it seem as though empirical Bayes is completely useless. After all, we do have some faith in our empirical Bayes example.


		Lines XX-XX:

		\begin{quotation}


		\end{quotation} 


	3. Include material from \emph{Appendix 7} in the main text [[done]]

		\begin{refquote}
		
			My final lingering concern is that something like appendix 7 seems important and should be included in the text? Was this done because of space limitations? I think the authors can appeal to the editor that the paper merits the extra space. As a compromise there could to be atleast a paragraph on what can go wrong and what that looks like. I think a figure would be useful here. If the informed/empirical priors are too narrow, the data will not really move the posterior. The authors certainly acknowledge it in various places, but it needs to be clearer to the reader.

		\end{refquote}


		\textbf{R:} This material was indeed placed in an appendix because of space constraints. We have added a figure showing the posterior distributions obtained from both priors and a paragraph summarizing \emph{Appendix 7}. As we do not wish to be too demanding with regard to extra space (and our manuscript was essentially at the standard length limit after the last revision), we hope that these smaller additions will be enough to emphasize the potential for spurious results after poor prior choices. Also note that the results for the galler-parasitoid community were somewhat similar. In order to more clearly demonstrate the potentially large effects of prior choice, we therefore showcase the results from \emph{Salix}-galler networks in this paragraph.


		Lines XX-XX:

		\begin{quotation}

		  Note that, as in all Bayesian analyses, our results do depend on the prior chosen. To demonstrate this, we repeated our analyses using a prior derived from a study of gallers found on several genotypes of \emph{Salix hookeriana} and the parasites which emerged from them (\citealp{Barbour2016,Barbour2016Dryad}; \emph{Appendix S7}). Although the study system is similar to that in~\citet{Kopelke2017}, the network is quite different due to using different genotypes of a single \emph{Salix} species rather than several \emph{Salix} species as the basis for sampling. While this had a relatively small effect on our expectations for the galler-parasitoid community (\emph{Appendix S7}), the prior based on \citep{Barbour2016} resulted in very high probabilities of interaction between \emph{Salix} and galler pairs that were not observed interacting (Fig.~\ref{prior_comparison}). No amount of additional sampling would allow us to conclude that a given \emph{Salix}-galler pair did not interact with a threshold interaction probability of 0.01. This is reasonable for the situation described in~\citet{Barbour2016}, as it is very likely that gallers which can interact with one \emph{S. hookeriana} genotype can interact with most others, but is not reasonable for the more diverse community in~\citet{Kopelke2017}. This demonstrates the importance of conducting a "sanity check" on the posterior distribution obtained from any given prior.

		\end{quotation}



	3. Reduce citations by Graham and Weinstein [[done]]

		\begin{refquote}

			To be perfectly honest, I feel more than a bit uncomfortable with the amount that Graham and Weinstein are cited in the introduction. I count 18 times. It would be more than sufficient to cite the work in the appropriate places and move on. I understand that the reviewers asked for a broader sense of the literature, but I think the authors have gone a bit overboard. I don’t know how often a reviewer asks the authors to cite himself less, but I assure you the reason that the paper is improved is not because I am cited nearly twenty times.


		\end{refquote}

		\textbf{R:} We have removed several of the citations to Graham and Weinstein/Weinstein and Graham, and thank the Reviewer for the opportunity to reduce our word count slightly. We had added particularly many citations of these papers because a couple of the Reviewers mentioned Graham and Weinstein as references that we had missed, and so we hope that the reduced number of references in the present revision still does justice to your work.



\clearpage

    \bibliographystyle{ecol_let} 
    \bibliography{manual} % Abbreviate journal titles.



\end{document}